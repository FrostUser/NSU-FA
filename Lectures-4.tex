\documentclass[12pt]{article}

\usepackage[utf8]{inputenc}
\usepackage[russian]{babel}

\usepackage{amssymb}
\usepackage{amsmath}
\usepackage{amsthm}
\usepackage{xcolor}

\usepackage{indentfirst}

%\usepackage[normalem]{ulem} % for crossing text out - \sout

% Redefining \def is impossible. I tried, but it is impossible.
%\let\def_prev\def

%%%%%%%%%%%%%%%%%%%%%%%%%%%%%%%%%%%%%%%%%%%%%%%
%           MATH OPERATORS SPACING            %
%%%%%%%%%%%%%%%%%%%%%%%%%%%%%%%%%%%%%%%%%%%%%%%

\let\existstemp\exists
\let\foralltemp\forall
\renewcommand{\exists}{\: \existstemp \:}
\newcommand{\existsonly}{\: \existstemp ! \:}
\renewcommand{\forall}{\: \foralltemp \:}

%%%%%%%%%%%%%%%%%%%%%%%%%%%%%%%%%%%%%%%%%%%%%%%
%            COMMAND SHORTHANDS               %
%%%%%%%%%%%%%%%%%%%%%%%%%%%%%%%%%%%%%%%%%%%%%%%

\newcommand{\example}{{\itshape Пример. }}
\newcommand{\equals}{\Leftrightarrow}
%\newcommand{\defi}{{\itshape Определение. }}
\newcommand{\exc}{{\bfseries Упражнение. }}
\newcommand{\norm}[1]{\| #1 \|}
\newcommand{\scal}[2]{\, < \!\! #1, #2 \!\! > \,}

\renewcommand{\leq}{\leqslant}
\renewcommand{\geq}{\geqslant}

%%%%%%%%%%%%%%%%%%%%%%%%%%%%%%%%%%%%%%%%%%%%%%%
%         THEOREM DEFINITION LINES            %
%%%%%%%%%%%%%%%%%%%%%%%%%%%%%%%%%%%%%%%%%%%%%%%

\newtheorem{lem}{Лемма}[section]
\newtheorem{defi}{Определение}[section]
\newtheorem{theorem}{Теорема}[section] % statement
\newtheorem{state}{Утверждение}[section] % statement

\setcounter{section}{3}

\begin{document}
	\title{Основы функционального анализа, второй семестр.}
	\author{Тресков Сергей Андреевич}
	\maketitle
	
	\section{Лекция}
	
	Ранее мы рассматривали пространства полиномов, определенные на интервале $(a, b)$ с весовой функцией $h(x) > 0$:
	
	$$\mathbb{L}_2^h (a, b): \int_a^b f^2(t) h(t) < \infty$$
	$$\scal{f}{g} \overset{df}{=} \int_a^b f(t) g(t) h(t) dt$$
	
	В прошлой лекции было введено обозначение $q_n(x)$ --- ортогональный полином $n$-ой степени. Озвучим ранее упомянутые
	свойства $q_n$ и дополним их новыми:
	
	\begin{itemize}
		\item При определении старшего коэффициента $a_i = 1$, $q_i$ одноначны.
		\item $\forall n > m, P_m \perp q_n$, где $P_m$ --- линейная комбинация $q_0, \dots , q_m$.
		\item Имеет место следующая рекуррентная формула.
		$$x \cdot q_n(x) = 
		\frac{a_n}{a_{n+1}} q_{n+1}(x) + (\frac{b_n}{a_n} - \frac{b_{n+1}}{a_{n+1}}) q_n(x) + \frac{a_{n-1}}{a_n} q_{n-1}(x)$$
		\item $\forall i$ Все нули $q_i(x)$ лежат на $(a, b)$.
		\item Все элементы $q_0 \dots q_n$ не имеют общих корней $\Leftrightarrow |q_n| + |q_{n+1}| > 0$
		\item В корне многочлена $n$-ой степени соседние полиномы имеют разные знаки.
	\end{itemize}
	
	Для доказательства этого факта воспользуемся выведенной рекуррентной формулой, учитывая, что $q(x_0) = 0$:
	$$ {\color{gray}x_0 \cdot 0} = 
	\frac{a_n}{a_{n+1}} q_{n+1}(x) + {\color{gray}(\frac{b_n}{a_n} - \frac{b_{n+1}}{a_{n+1}}) \cdot 0} + \frac{a_{n-1}}{a_n} q_{n-1}(x)$$
		
	%Здесь также нужно дописать про корни полинома. Вероятно стоит взять из Александрова - я малость упустил.%
	
	\subsection*{Классические ортогональные многочлены}

	В данном курсе рассматриваются следующие полиномы:

	\begin{table}[!th]
		\begin{tabular}{|l|l|l|l|}
			\hline
			Название & Обозначение & Интервал ортогональности & Весовая функция \\
			\hline
			Эрмитовы & $H_n(x)$ & $\mathbb{R}$ & $e^{-x^2}$ \\
			Лагерра  & $L_n(x)$ & $\mathbb{R}_+$ & $e^{-x}$ \\
			Лежандра & $P_n(x)$ & $(-1, 1)$ & $1$ \\
			Чебышёва & $T_n(x)$ & $(-1, 1)$ & $\frac{1}{\sqrt{1-x^2}}$ \\
			\hline
		\end{tabular}
	\end{table}
	
	%ортонормированы - это ведь краткое причастие и пишется с одной "н", правильно?
	Как ни печально, все рассматриваемые здесь многочлены ортогональны, но не ортонормированы.
	Стоит отметить, что в таблице указаны только многолены Чебышёва первого рода. Многочлены Чебышёва второго рода в курсе 
	не рассматриваются.
	
	\subsubsection{Эрмитовы многочлены}
	
	Перед тем, как приступить к рассмотрению формулы для получения эрмитова многочлена $n$-ой степени, называемой 
	\textbf{формулой Родрига}, рассмотрим функцию $\varphi(x) = e^{-x^2}$. Продифференцировав её, получаем 
	$\varphi'(x) = -2x \varphi(x)$. Таким образом, нетрудно убедиться, что
	$$H_n(x) \overset{df}{=} (-1)^n e^{x^2} \varphi^{(n)} (x)$$
	является полиномом $n$-ой степени. При этом коэффициент при старшей степени равняется $2^n$.
	
	Теперь докажем, что $H_n$ ортогональны. Для этого рассмотрим скалярное произведение $H_n$ и $H_m$, где $n > m$.
	$$\scal{H_n}{H_n} = \int_{-\infty}^{\infty} H_n(x) H_m(x) e^{-x^2} dx = (-1)^n \int_{-\infty}^{\infty} \varphi^{(n)}(x) H_m(x) dx$$
	<<Какое ваше первое желание, когда вы видите производную в интеграле?>> --- правильно, интегрировать по частям:
	$$(-1)^n \int_{-\infty}^{\infty} \varphi^{(n)}(x) H_m(x) dx = (-1)^n ( {\color{gray}(\varphi^{(n-1)}(x) H_m(x)) 
	\underset{-\infty}{\overset{\infty}{|}}} - (\int_{-\infty}^{\infty} \varphi^{(n-1)}(x) (H_m)' dx ) )$$
	Первое слагаемое уходит, так как $\varphi(x) \underset{x \rightarrow \infty}{\rightarrow} 0$. Это означает, что мы можем без проблем
	продифференцировать $m$ раз, так, что получится такое выражение:
	$$(-1)^{n-m} \int_{-\infty}^{\infty} \varphi^{(n-m)}(x) (H_m(x))^{(m)} dx$$
	В условиях $n > m$ $\varphi(x)^{(n-m)}$ интегрируема на $\mathbb{R}$ и $\int_{\mathbb{R}} \varphi(x)^{(n-m)} dx = 0$. С другой 
	стороны, так как функция $H_m$ является полиномом $m$-ой степени, то $(H_m)^{(m)}$ является константой. Значит окончательное выражение
	равно нулю и получаем, что
	$$n \neq m \scal{H_n}{H_m} = 0$$
	Если же предположить, что $n=m$, то, применяя такие же шаги, что и в прошлых вычислениях, получаем интеграл Пуассона, который 
	равняется $\sqrt{\pi}$.
	$$\scal{H_n}{H_n} = \int_{-\infty}^{\infty} H_n(x) H_n(x) dx = \dots = \int_{-\infty}^{\infty} e^{-x^2} dx = \sqrt{\pi}$$
	
	Таким образом, как и было сказано выше, эрмитовы многочлены ортогональны, но не ортонормированы.
	
	Попробуем получить рекуррентную формулу для эрмитовых многочленов, дифференцируя $\varphi^{(n)}$ и выражая ее через предыдущие
	производные.
	
	$$ \varphi^{(n+1)}(x) = -2x \cdot \varphi^{(n)}(x) - 2n \cdot \varphi^{(n-1)}(x) $$
	Домножив полученное уравнение на $(-1)^{n+1} e^{-x^2}$, получим искомое рекуррентное соотношение:
	$$H_{n+1} (x) = 2x \cdot H_n(x) - 2n \cdot H_{n-1} (x)$$
	
	Мы рассмотрели уже два способа задания эрмитовых ортогональных многочленов: через формулу Родрига и через рекуррентную формулу.
	Также очевидно получение полиномов через процесс ортогонализации Грама-Шмидта. Рассмотрим еще несколько способов.\\ \\
	
	Продифференцируем $H_n(x) = (-1)^n e^{x^2} \varphi^{(n)}(x)$. Получаем вот такое выражение:
	$$H_n'(x) = (-1)^n 2x e^{x^2} \varphi^{(n)}(x) + (-1)^n e^{x^2} \varphi^{n+1}(x) = 2x H_n(x) - H_{n+1}(x)$$
	Используя рекуррентную формулу для эрмитовых многочленов, данное равенство преобразуется в дифференциальное уравнение:
	$$H_n'(x) = 2n H_{n-1}(x)$$
	
	\begin{defi}
	Для числовой последовательности $a_0 \dots a_n$ можно ввести формальный степенной ряд $f(t) = \sum_{n=0}^{\infty} a_n t^n$ 
	--- \textbf{производящую функцию} этой последовательности.
	\end{defi}
	
	Для задания эрмитовых многочленов также можно ввести производящую функцию:
\end{document}