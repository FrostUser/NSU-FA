\documentclass[12pt]{article}

\usepackage[utf8]{inputenc}
\usepackage[russian]{babel}

\usepackage{amssymb}
\usepackage{amsmath}
\usepackage{amsthm}
\usepackage{xcolor}

\usepackage{indentfirst}

%\usepackage[normalem]{ulem} % for crossing text out - \sout

% Redefining \def is impossible. I tried, but it is impossible.
%\let\def_prev\def

%%%%%%%%%%%%%%%%%%%%%%%%%%%%%%%%%%%%%%%%%%%%%%%
%           MATH OPERATORS SPACING            %
%%%%%%%%%%%%%%%%%%%%%%%%%%%%%%%%%%%%%%%%%%%%%%%

\let\existstemp\exists
\let\foralltemp\forall
\renewcommand{\exists}{\: \existstemp \:}
\newcommand{\existsonly}{\: \existstemp ! \:}
\renewcommand{\forall}{\: \foralltemp \:}

%%%%%%%%%%%%%%%%%%%%%%%%%%%%%%%%%%%%%%%%%%%%%%%
%            COMMAND SHORTHANDS               %
%%%%%%%%%%%%%%%%%%%%%%%%%%%%%%%%%%%%%%%%%%%%%%%

\newcommand{\example}{{\itshape Пример. }}
\newcommand{\equals}{\Leftrightarrow}
%\newcommand{\defi}{{\itshape Определение. }}
\newcommand{\exc}{{\bfseries Упражнение. }}
\newcommand{\norm}[1]{\| #1 \|}
\newcommand{\scal}[2]{\, < \!\! #1, #2 \!\! > \,}

\renewcommand{\leq}{\leqslant}
\renewcommand{\geq}{\geqslant}

%%%%%%%%%%%%%%%%%%%%%%%%%%%%%%%%%%%%%%%%%%%%%%%
%         THEOREM DEFINITION LINES            %
%%%%%%%%%%%%%%%%%%%%%%%%%%%%%%%%%%%%%%%%%%%%%%%

\newtheorem{lem}{Лемма}[section]
\newtheorem{defi}{Определение}[section]
\newtheorem{theorem}{Теорема}[section] % statement
\newtheorem{state}{Утверждение}[section] % statement

\setcounter{section}{2}

\begin{document}
	\title{Основы функционального анализа, второй семестр.}
	\author{Тресков Сергей Андреевич}
	\maketitle
	
	\section{Лекция}
	
	Перейдем к основному содержанию нашего курса --- гильбертовым пространствам. Пространствами, рассматриваемыми 
	в дальнейших лекциях будут:
	\begin{itemize}
		\item $\mathbb{L}_2(\mathbb{X})$ \\
		$\scal{f}{g} \overset{df}{=} \int_{\mathbb{X}} f$ \\
		$\norm{f}_2 = \sqrt{\int_{\mathbb{X}} |f|^2}$ \\
		\exc Доказать полноту $\mathbb{L}_2$.
		
		\item $l_2(\mathbb{X})$ \\
		В первой и второй лекциях свойства этого пространства уже были подробным образом рассмотрены. \\
		\exc Доказать полноту $l_2$.
	\end{itemize}
	Пространства $l_2$ и $\mathbb{L}_2$ ---полные, сепарабельные и бесконечномерные пространства.
	
	В прошлой лекции было рассмотрено неравенство Бесселя: \\
	$$\norm{h}^2 \geq \scal{h_k}{e_k}$$
	Где $h = (\alpha_1, \alpha_2, \ldots, \alpha_n, \ldots)$ и, так как система векторов $\{ e_i \}$ ортонормированна, 
	$\alpha_k = \scal{h_k}{e_k}$. Вектор h относительно некоторого подпространства может быть представлен в виде суммы
	ортогональной проекции g и ортогонального дополнения f к этому подпространству. 
	$$h = g + f$$
	Если h лежит в замыкании линейной оболички $\{ e_i \}$, то ортогональное дополнение $f = 0$, что будет означать 
	$\norm{h} = \norm{g}$, вследствие чего неравенство Бесселя обращатеся в равенство, которое зачастую наывают 
	\textbf{равенством Парсеваля} ($\norm{h}^2 = \sum_1^{\infty} |\alpha_i|^2$).\\
	Оно несколько отличается от одноименного равенства из рядов Фурье. Рассматривая скалярное произведение, можно получить
	намного более похожее равенство:
	$$ \scal{x}{y} = \sum \alpha_i \overline{\beta_i} $$
	Введем $x_n = \sum_1^n \alpha_i e_i$, тогда:
	$$ \scal{x_n}{y} = \sum_1^n \alpha_i \cdot \! \scal{e_i}{y} = \sum_1^n \alpha_i \overline{\beta_i}$$
	При $n \rightarrow \infty$ данное выражение стремится к:
	$$ \scal{x}{y} = \sum_1^{\infty} \alpha_i \overline{\beta_i} $$
	
	Если бесконечномерное пространство H --- сепарабельное, то в нём найдется счётный набор векторов.
	
	Рассмотрим последовательность $\vec{h_i}$. Вычеркнем из нее те вектора, которые являются линейной комбинацией предыдущих.
	Получим линейно независимый набор векторов и применим процесс ортогонализации Грама-Шмидта. В итоге получим полную счётную 
	последовательонсть.
	
	Если у нас есть набор ортонормированных векторов $\{ e_u \}$ и $\alpha \in l_2$, то ряд $\sum \alpha_i e_i$ сходится по 
	критерию Коши, так как квадрат разности частичных сумм оценивается неравенством Бесселя.
	
	По сути, приведенные выше утверждения составляют \textbf{теорему Рисса --- Фишера}:
	\begin{theorem}
		Пусть $x_1, \dots ,x_n, \dots $-- произвольная ортонормированная система векторов в гильбертовом пространстве H, и пусть 
		числа \\
		$\lambda _1, \dots ,\lambda _n, \dots $ таковы, что ряд $\sum |\lambda_n|^2$ сходится. Тогда существует такой 
		вектор $x\in H$, что $\lambda _n=(x,x_n)$ и
		\begin{displaymath}
			\vert\vert x\vert\vert ^2=\sum_{n=1}^{\infty } \vert\lambda _n\vert^2,
		\end{displaymath}
		т.е. такой x, для которого $\lambda _n$ являются коэффициентами Фурье, а норма вычисляется в 
		соответствии с равенством Парсеваля. 
	\end{theorem}
	{\color{red} Теорема Рисса --- Фишера не была сказана на лекциях, так что, надеюсь, на экзамене её не будет.}
	
	В итоге, нами было получено, что любое сепарабельное гильбертово пространство изоморфно $l_2$. \\
	На всякий случай <<освежим в памяти>> определение измоморизма.
	\begin{defi}
		Два множества называются \textbf{изоморфными}, если существет обратомое линейное отображение, такое что скалярное
		произведение переходит в скалярное произведение.
	\end{defi}
	
	Перед тем как перейти к рассмотрению конкретных ортонормированных систем функций, введем понятие гильбертова базиса:
	\begin{defi}
		$\{ e_i \}$ --- ортонормировання система векторов, называется \textbf{гильбертовым базисом}, если любой вектор
		пространства может быть представлен в виде бесконечной линейной комбинации $ \{ e_i \} $.
	\end{defi}
	
	{\color{gray} Гильбертов базис отличается от обычного словом <<бесконечной>>.}
	
	\begin{state}
		В сепарабельном гильбертовом пространстве найдется счтётный гильбертов базис.
	\end{state}
	
	\example $\mathbb{L}_2 (0; 2\pi)$, $\{ \frac{1}{\sqrt{2\pi}} \cdot e^{int} \}$, $n \in \mathbb{Z}$ --- ортонормированная система.
	{\color{gray} В чем, разумеется, вы легко убедитесь.}\\
	Докажем, что эта система функций полна.\\
	\textbf{Идея доказательства}: в сущности, требуется доказать, что, если существует такая f, которая $\perp \vec{e_n}$,
	для любых n, то тогда $f \equiv 0$, что доказывает полноту $\{ e_n \}$
	\begin{proof}
		Так как функция $f$ ортогональна всем векторам из нашего базиса, можем записать:
		$$ \forall n, \int_0^{2\pi} f(t) \cdot e^{-int} = 0 $$
		Введем $F(x) = \int_0^x f(t) dt$. Проинтегрируем равенство по частям:
		$$ 0 = F(x) \cdot e^{-int} \underset{0}{\overset{{2\pi}}{|}} + n \cdot \int_0^{2\pi} F(t) \cdot e^{-int} dt $$
		Отсюда получаем, что $\int_0^{2\pi} (F(t) + C) \cdot e^{-int} dt = 0$, для $n \neq 0$. \\
		Рассмотрим функцию $\Phi(t) = F(t) + C$ --- определена на интервале $(0, 2\pi)$ и непрерывна.
		Отсюда, по теореме Фейера, можем найти тригонометрический полином, приближающий данную функцию:
		$$\forall \varepsilon > 0 \exists \sum_{-n}^n \alpha_k e^{ikt}$$\
		при этом:
		$$ \underset{t \in (0,2\pi)}{sup} | (\sum_{-n}^n (\alpha_k e^{ikt})) - \Phi(t)| < \varepsilon $$
		При этом, каждый такой моном $\alpha_k e^{ikt}$ ортогонален $\Phi$, такую уж функцию мы выбрали.
		
		$$ \norm{\Phi(t)}^2_{\mathbb{L}_2} = \int_0^{2\pi} \Phi(t) \cdot \overline{\Phi(t)} dt \leq $$
		$$ \leq \varepsilon \int_0^{2\pi} | \Phi(y) | dt \leq \norm{\Phi} \cdot \sqrt{2\pi} \varepsilon $$
		Отсюда получается, что $\norm{\Phi} \leq \varepsilon \sqrt{2\pi}$, значит $\norm{\Phi} \rightarrow 0$ или, точнее сказать,
		равна нулю почти всюду.\\
	\end{proof}
	
	К сожалению, в доказательстве, которое мы привели, есть несколько <<узких мест>>:
	\begin{enumerate}
		\item Не доказано, что, если $\int_0^{2\pi} f(t) \equiv 0$, то $f(t) = 0$ почти всюду.
		\item Не сказано, что $f \in \mathbb{L}_1$, казано лишь про $f \in \mathbb{L}_2$. \\
		В прошлом семестре было сказано, что если $f \in \mathbb{L}_2$ на множестве конечной меры, то $f \in \mathbb{L}_2$.
		\item Не доказана <<правомерность>> интегрирования по частям. \\
		Для доказательства этого используется утвержение о том, что $f$ может быть приближена гладкими функциями, а
		гладкиие функции плотны в $\mathbb{L}_2$.
	\end{enumerate}
	
	Начиная с этого момент, мы будем рассматривать не унитарные, а только евклидовы (вещественные) гильбертовы пространства.
	
	\subsection{Классические многочлены}
	По теореме Вейерштрасса любая непрерывная функция на ограниченном промежутке может быть приближена полиномом. \\
	Так как гладкие функции плотны в $\mathbb{L}_2$, то многочлены тоже плотны в $\mathbb{L}_2$.
	
	Пусть существет промежуток (a,b) - не обязательно ограниченный и рассмотрим $h(t) > 0$, а также пространство
	$\mathbb{L}_2^h (a,b)$ --- функции, такие, что $\int_a^b |f(t)|^2 h(t) dt < \infty$.
	Это пространство является евклидовым, если определено такое скалярное произведение:
	
	$$ \scal{f}{g} = \int_a^b f(t) g(t) g(t) dt $$
	
	Если берём $ \mathbb{L}_2^h (-1, 1)$ то можно взять $h \equiv 1$, так как тогда система функций,
	$1, x, x^2, \ldots$, испольуя процесс ортогонализации Грама-Шмидта, преобразуется в $q_1, q_2, q_3, \ldots$ --- ортонормированную 
	систему. При этом, $q_{n+1}$ восстанавливается из $ \{ q_n \} $ двумя способами --- ортогональное дополнение можно выбрать либо
	со старшим коэффициентом $a_{n+1} = 1$, либо с $a_{n+1} = -1$. Условимся, что старший коэффициент при таком раскладе всегда будем 
	выбирать равным единице.
	
	Тогда можно определить следующие свойства:
	\begin{itemize}
		\item Одноначность (при принятых выше условиях).
		\item Если $n > m$, тогда $q_n \perp \mathtt{P}_m$
		\item Имеет место рекуррентное соотношение: \\
		$$ x \cdot q_n(x) = 
		\alpha_{n+1,n} q_{n+1} + \alpha_{n,n} q_n(x) + \alpha_{n-1, n} q_{n-1} (x) + \ldots + \alpha_{n-k, n} q_{n-k} (x) $$
		
		Где $\alpha_{k, n}$ - некоторые коэффициенты, с $k$ --- степенью текущего слагаемого и $n$ --- степень многочлена, который 
		домножается	на $x$.
		
		Рассмотрим $\scal{x q_n, q_k(x)} = \int_a^b x q_n(x) q_k(x) dx = \alpha_{k,n} \int q_k(x) q_k(x) = \alpha_{k,n}$.
		Отсюда следует, что $\alpha_{k,n} = \alpha_{n,k} \Rightarrow$ для $n > k+1,\alpha_{k,n} = 0$. Это означает, что исходное соотношение
		может быть записано как:
		$$ x \cdot q_n(x) = 
		\alpha_{n+1,n} q_{n+1} + \alpha_{n,n} q_n(x) + \alpha_{n-1, n} q_{n-1} (x)$$
		Запишем $q_n = a_n x^n + b_n x^{n-1} + \ldots$
		Тогда $\alpha_{n+1, n} = \frac{a_n}{a_{n+1}}$ и, в силу равенства $\alpha_{k,n} = \alpha_{n,k}$, получаем $\alpha_{n-1, n} 
		= \frac{a_{n-1}}{a_n}$. В рассматриваемом рекуррентном соотношении, коэффициенты при $x_n$ должны быть равны с обеих
		сторон, что означает $b_n = \alpha_{n+1, n} b_{n+1} + \alpha_{n,n} a_n = \frac{a_n}{a_{n+1}} b_{n+1} + \alpha_{n,n} a_n$, 
		откуда следует $\alpha_{n,n} = \frac{b_n}{a_n} - \frac{b_{n+1}}{a_{n+1}}$. Теперь, все коэффициенты данного соотношения
		найдены:
		$$ x \cdot q_n(x) = 
		\frac{a_n}{a_{n+1}} q_{n+1} + (\frac{b_n}{a_n} - \frac{b_{n+1}}{a_{n+1}}) q_n(x) + \frac{a_{n-1}}{a_n} q_{n-1} (x)$$
	\end{itemize}
	
	\begin{state}
		Все ортогональные многочлены степени n имеют ровно n корней, причем эти корни:
		\begin{enumerate}
			\item $x_i \in \mathbb{R}$
			\item $x_i$ --- простые корни.
			\item $x_i \in (a,b)$
		\end{enumerate}
		\begin{proof}
			Используем метод <<от противного>>: \\
			Пусть существует только $k < n$ корней в $(a,b)$. Считаем k корней, где $q_n(x)$ меняет знак. Рассмотрим 
			$\mathtt{P}_k = (x - x_1) \cdots (x - x_k)$. Тогда $q_n(x) \mathtt{P}_k(x)$ сохраняет знак. 
			$$ \int_a^b \mathtt{P}_k (x) q_n(x) h(x) \neq 0$$, что противоречит предположению $k < n$. \\
			Отсюда следует $k = n$, а это и означает, что все корни многочлена $q_n$ расположены на интервале $(a,b)$ 
			и различны.\\
		\end{proof}
	\end{state}
\end{document}
