\documentclass[12pt]{article}

\usepackage[utf8]{inputenc}
\usepackage[russian]{babel}

\usepackage{amssymb}
\usepackage{amsmath}
\usepackage{amsthm}
\usepackage{xcolor}

\usepackage{indentfirst}

%\usepackage[normalem]{ulem} % for crossing text out - \sout

% Redefining \def is impossible. I tried, but it is impossible.
%\let\def_prev\def

%%%%%%%%%%%%%%%%%%%%%%%%%%%%%%%%%%%%%%%%%%%%%%%
%           MATH OPERATORS SPACING            %
%%%%%%%%%%%%%%%%%%%%%%%%%%%%%%%%%%%%%%%%%%%%%%%

\let\existstemp\exists
\let\foralltemp\forall
\renewcommand{\exists}{\: \existstemp \:}
\newcommand{\existsonly}{\: \existstemp ! \:}
\renewcommand{\forall}{\: \foralltemp \:}

%%%%%%%%%%%%%%%%%%%%%%%%%%%%%%%%%%%%%%%%%%%%%%%
%            COMMAND SHORTHANDS               %
%%%%%%%%%%%%%%%%%%%%%%%%%%%%%%%%%%%%%%%%%%%%%%%

\newcommand{\example}{{\itshape Пример. }}
\newcommand{\equals}{\Leftrightarrow}
%\newcommand{\defi}{{\itshape Определение. }}
\newcommand{\exc}{{\bfseries Упражнение. }}
\newcommand{\norm}[1]{\| #1 \|}
\newcommand{\scal}[2]{<#1, #2>}

\renewcommand{\leq}{\leqslant}
\renewcommand{\geq}{\geqslant}

%%%%%%%%%%%%%%%%%%%%%%%%%%%%%%%%%%%%%%%%%%%%%%%
%         THEOREM DEFINITION LINES            %
%%%%%%%%%%%%%%%%%%%%%%%%%%%%%%%%%%%%%%%%%%%%%%%

\newtheorem{lem}{Лемма}[section]
\newtheorem{defi}{Определение}[section]
\newtheorem{theorem}{Теорема}[section] % statement
\newtheorem{state}{Утверждение}[section] % statement

\setcounter{section}{2}

\begin{document}
	\title{Основы функционального анализа, второй семестр.}
	\author{Тресков Сергей Андреевич}
	\maketitle
	
	\section{Лекция}
	
	Перейдем к основному содержанию нашего курса --- гильбертовым пространствам. Пространствами, рассматриваемыми 
	в дальнейших лекциях будут:
	\begin{itemize}
		\item $\mathbb{L}_2(\mathbb{X})$ \\
		$\scal{f}{g} \overset{df}{=} \int_{\mathbb{X}} f$ \\
		$\norm{f}_2 = \sqrt{\int_{\mathbb{X}} |f|^2}$ \\
		\exc Доказать полноту $\mathbb{L}_2$.
		
		\item $l_2(\mathbb{X})$ \\
		В первой и второй лекциях свойства этого пространства уже были подробным образом рассмотрены. \\
		\exc Доказать полноту $l_2$.
	\end{itemize}
	Пространства $l_2$ и $\mathbb{L}_2$ ---полные, сепарабельные и бесконечномерные пространства.
	
	В прошлой лекции было рассмотрено неравенство Бесселя: \\
	$$\norm{h}^2 \geq \scal{h_k}{e_k}$$
	Где $h = (\alpha_1, \alpha_2, \ldots, \alpha_n, \ldots)$ и, так как система векторов $\{ e_i \}$ ортонормированна, 
	$\alpha_k = \scal{h_k}{e_k}$. Вектор h относительно некоторого подпространства может быть представлен в виде суммы
	ортогональной проекции g и ортогонального дополнения f к этому подпространству. 
	$$h = g + f$$
	Если h лежит в замыкании линейной оболички $\{ e_i \}$, то ортогональное дополнение $f = 0$, что будет означать 
	$\norm{h} = \norm{g}$, вследствие чего неравенство Бесселя обращатеся в равенство, которое зачастую наывают 
	\textbf{равенством Парсеваля} ($\norm{h}^2 = \sum_1^{\inf} |\alpha_i|^2$).\\
	Оно несколько отличается от одноименного равенства из рядов Фурье. Рассматривая скалярное произведение, можно получить
	намного более похожее равенство:
	$$ \scal{x}{y} = \sum \alpha_i \overline{\beta_i} $$
	Введем $x_n = \sum_1^n \alpha_i e_i$, тогда:
	$$ \scal{x_n}{y} = \sum_1^n \alpha_i \cdot \scal{e_i}{y} = \sum_1^n \alpha_i \overline{\beta_i}$$
	При $n \rightarrow \inf$ данное выражение стремится к:
	$$ \scal{x}{y} = \sum_1^{\inf} \alpha_i \overline{\beta_i} $$
	
	Если бесконечномерное пространство H --- сепарабельное, то в нём найдется счётный набор векторов.
	
	Рассмотрим последовательность $\vec{h_i}$. Вычеркнем из нее те вектора, которые являются линейной комбинацией предыдущих.
	Получим линейно независимый набор векторов и применим процесс ортогонализации Грама-Шмидта. В итоге получим полную счётную 
	последовательонсть.
	
	Если у нас есть набор ортонормированных векторов $\{ e_u \}$ и $\alpha \in l_2$, то ряд $\sum \alpha_i e_i$ сходится по 
	критериюю Коши, так как квадрат разности частичных сумм оценивается неравенством Бесселя.
	
	По сути, приведенные выше утверждения составляют \textbf{теорему Рисса --- Фишера}:
	\begin{theorem}
		Пусть $x_1, \dots ,x_n, \dots $-- произвольная ортонормированная система векторов в гильбертовом пространстве H, и пусть 
		числа $\lambda _1, \dots ,\lambda _n, \dots $таковы, что ряд $\sum |\lambda_n|^2$ сходится. Тогда существует такой 
		вектор $x\in H$, что $\lambda _n=(x,x_n)$ и
		\begin{displaymath}
			\vert\vert x\vert\vert ^2=\sum_{n=1}^{\infty } \vert\lambda _n\vert^2,
		\end{displaymath}
		т.е. такой x, для которого $\lambda _n$ являются коэффициентами Фурье, а норма вычисляется в 
		соответствии с равенством Парсеваля. 
	\end{theorem}
	{\color{red} Теорема Рисса --- Фишера не была сказана на лекциях, так что ,надеюсь, на экзамене ее не будет.}
	
	В итоге, нами было получено, что любое сепарабельное гильбертово пространство изоморфно $l_2$. \\
	На всякий случай <<освежим в памяти>> определение измоморизма.
	\begin{defi}
		Два множества называются \textbf{изоморфными}, если существет обратомое линейное отображение, такое что скалярное
		произведение переходит в скалярное произведение.
	\end{defi}
	
	Перед тем как перейти к рассмотрению конкретных ортонормированных систем функций, введем понятие гильбертова базиса:
	\begin{defi}
		$\{ e_i \}$ --- ортонормировання система векторов, называется \textbf{гильбертовым базисом}, если любой вектор
		пространства может быть представлен в виде бесконечной линейной комбинации $ \{ e_i \} $.
	\end{defi}
	
	{\color{gray} Гильбертов базис отличается от обычного словом <<бесконечной>>.}
	
	\begin{state}
		В сепарабельном гильбертовом пространстве найдется счтётный гильбертов базис.
	\end{state}
	
	\example $\mathbb{L}_2 (0; 2\pi)$, $\{ \frac{1}{\sqrt{2\pi}} \cdot e^{int} \}$, $n \in \mathbb{Z}$ --- ортонормированная система.
	{\color{gray} В чем, разумеется, вы легко убедитесь.}\\
	Докажем, что эта система функций полна.\\
	\textbf{Идея доказательства}: .\\
\end{document}