\documentclass[12pt]{article}

\usepackage[utf8]{inputenc}
\usepackage[russian]{babel}

\usepackage{amssymb}
\usepackage{xcolor}

\usepackage{indentfirst}

% Redefining \def is impossible. I tried, but it is impossible.
%\let\def_prev\def

\newcommand{\example}{{\itshape Пример. }}
\newcommand{\equals}{\Leftrightarrow}
\newcommand{\defi}{{\itshape Определение. }}
\newcommand{\exc}{{\bfseries Упражнение. }}

\renewcommand{\leq}{\leqslant}
\renewcommand{\geq}{\geqslant}


\begin{document}
	\title{Основы функционального анализа, второй семестр.}
	\author{Тресков Сергей Андреевич}
	\maketitle
	
	\section{Векторные пространства}
	
	\defi \textbf{Векторное пространство} -- это математическая структура, которая формируется набором элементов, называемых векторами, для 
	которых 	определены операции сложения друг с другом и умножения на число -- скаляр.
	
	Ранее в курсе математического анализа рассматривались векторные пространства $\mathbb{R}^n$ -- векторное пространство вещественных
	чисел, $\mathbb{C}^n$ -- векторное пространство комлексных чисел, C[a,b] -- пространство непрерывных функций, $\mathbb{L}_1$ -- 
	пространство интегрируемых функций, $\mathbb{L}_2$ -- пространство функций, интегрируемых с квадратом.
	
	В данном курсе преимущественно рассматривается векторное пространство $l_2$.
	$$l_2 : x = (x_1, x_2, ...)$$
	$$\sum_{i=1} |x_i|^2 < \infty$$

	Стоит заметить, что в отличие от ранее рассматриваемых векторных пространств, пространство $l_2$ не имеет конечной размерности.
	
	\defi Произвольное множество векторов из векторного пространства E, называется \textbf{линейно независимым}, если каждое конечное
	подмножество векторов, лежащее в E, тоже линейно независимо.
	
	Для пространства $l_2$ потребуется ввести новое определение предела, используя понятие метрики:
	
	\defi \textbf{Метрическим пространством} называется пара (M, $\rho$), где M -- множество, $\rho$ -- функция $\rho : M^2 \rightarrow 
	\mathbb{R}$, причем для $\rho$ выполняются следующие условия:
	\begin{enumerate}
		\item $\rho(x,y) \geq 0 \& (\rho = 0 \equals x=y)$
		\item $\rho(x,y) = \rho(y,x)$
		\item $\rho(x,z) \leq \rho(x,y) + \rho(y,z)$ (неравенство треугольника)
	\end{enumerate}
	
	Так же как и в курсе математического анализа, введем определение открытой элементарной окрестности:
	$$B(x, \epsilon) = \{y \in M | \rho(x,y) < \epsilon\}$$
	
	Таким образом, возможно определение предела как в терминах метрик, так и в терминах окрестностей. Здесь приводится первое определение,
	а второе остается в качестве упражнения:
	$$f : M_1 \rightarrow M_2, lim_{x \rightarrow x_0} f(x) = y \in M_2 : \forall \epsilon > 0 \exists \delta > 0, \forall x \neq x_0$$
	$$\rho_1(x, x_0) < \delta \Rightarrow \rho_2(f(x), y) > \epsilon$$
	
	\defi Назовем подмножество $U \subset M$ \textbf{открытым}, если любая точка в нем содержится вместе с некоторой окрестностью.
	
	\example Рассмотрим дискретное метрическое пространство:
	$$\rho = \{^{1, x \neq y}_{0, x = y}$$
	Любое подмножество, содержащееся в пространстве с такой метрикой является открытым: каждая точка содержит окрестность радиуса 
	$\frac{1}{2}$.
	\defi Множество M называется \textbf{замкнутым}, если оно содержит все свои предельные точки.
	
	\defi Замыкание множества -- это объединение множества и всех его предельных точек. Обозначают $\bar{M}$ или Cl M.
	
	\exc Доказать, что замыкание множества является замкнутым.
	
	\exc Доказать, что дополнение к открытому множеству замкнуто, а к замкнутому открыто.
	
	\defi $M_1 \subset M$, $M_2 \subset M$. Подмножество $M_1$ плотно в $M_2 \equals \bar{M}_1 \subset M_2$
	
	\example Множество рациональных чисел плотно в множестве иррациональных.
	
	\defi Если $M_1 \subset M$, $\bar{M}_1 = M$, то $M_1$ всюду плотно. (То есть для любой точки из M существует последовательность
	из $M_1$, которая сходится к этой точке.)
	
	\defi Множество M называется сепарабельным, если у него найдется счетное, всюду плотное подмножество.
	
	\defi Счетное множество -- множество, которое можно пронумеровать.
	
	\defi Последовательность x называется фундаментальной, если
	$$\forall \epsilon > 0 \exists N(\epsilon), \forall n_1, n_2 > N, \rho(x_{n_1}, x_{n_2}) < \epsilon$$
	
	\defi Пространство называется полным, если любая содержащаяся в нем фундаментальная последовательность имеет предел.
	
	В любом векторном пространстве существует понятие длины или нормы вектора:
	$$||...|| : V \rightarrow \mathbb{R}$$
	\begin{enumerate}
		\item $||x|| \geq 0 (||x|| = 0 \equals x = 0)$
		\item $||\alpha x|| = |\alpha| ||x||$
		\item $||x + y|| \leq ||x|| + ||y||$
	\end{enumerate}
	
	Для нормированного пространства можно легко определить метрику, вводя $\rho(x,y) = ||x-y||$. При этом будут выполняться все аксиомы,
	определенные для метрики ранее.
	
	\defi Две нормы p и q на пространстве V называются \textbf{эквивалентными}, если существуют две положительные константы $C_1$ и $C_2$,
	такие, что для любого $x \in V$ выполняется 
	$$C_1 p(x) \leq q(x) \leq C_2 p(x)$$
	Эквивалентные нормы задают на пространстве одинаковую топологию. В конечномерном пространстве все нормы эквивалентны.
	
	\defi Банахово пространство -- полное линейное пространство.
	
	Значительное внимание в курсе уделено пространствами со скалярным произведением.
	$$<...,...> : E^2 \rightarrow C$$
	Для скалярного произведения определены следующие свойства:
	\begin{enumerate}
		\item $<x,y> = \bar{<y,x>}$
		\item $<\alpha x_1 + \beta x_2, y> = \alpha <x_1, y> + \beta <x+2, y>$
		\item $<x, x> \geq 0 (<x, x> = 0 \equals x = 0)$
	\end{enumerate}
	
	\defi $||x|| = \sqrt{<x, x>}$ {\color{gray} Пока это <<контрабандное>> утверждение, докажем его позже.}
	
	\exc Доказать тождество параллелограмма: 
	$$||x+y||^2 + ||x-y||^2 = 2 ||x||^2 + 2 ||y||^2$$
	Так как пространство $l_2$ - бесконечномерное пространство, для него потребуется доказать неравенство Шварца (оно же неравенство
	Коши-Буняковского):
	$$|<x, y>| \leq ||x|| \cdot ||y||$$
	Введем две вспомогательные переменные:
	$$\theta = \frac{<x, y>}{|<x, y>|}, t \in \mathbb{R}$$
	
	$$0 \leq ||\bar{\theta} x + t y||^2 = <\bar{\theta} x + t y, \bar{\theta} x + t y> = $$
	$$\bar{\theta} <x, \bar{\theta} x + t y> + t <y, \bar{\theta} x + t y> = $$
	$$|\theta|^2 ||x||^2 + t \bar{\theta} <x, y> + t \theta <y, x> + t^2 ||y||^2 = $$
	$$t^2 ||y||^2 + 2t |<x,y>| + ||x||^2$$
	Таким образом, получаем:
	$$|<x, y>|^2 \leq ||x||^2 ||y||^2$$
	И, взяв квадратный корень из обеих частей выражения, получим искомое неравенство.
	
	\defi \textbf{Гильбертовым пространством} называется полное пространство со скалярным произведением.
	
	\example $\mathbb{R}^n$ - гильбертово пространство.
\end{document}