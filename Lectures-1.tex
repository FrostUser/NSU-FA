\documentclass[12pt]{article}

\usepackage[utf8]{inputenc}
\usepackage[russian]{babel}

\usepackage{amssymb}
\usepackage{xcolor}

\usepackage{indentfirst}

\usepackage[normalem]{ulem} % for crossing text out - \sout

% Redefining \def is impossible. I tried, but it is impossible.
%\let\def_prev\def

\newcommand{\example}{{\itshape Пример. }}
\newcommand{\equals}{\Leftrightarrow}
\newcommand{\defi}{{\itshape Определение. }}
\newcommand{\exc}{{\bfseries Упражнение. }}

\renewcommand{\leq}{\leqslant}
\renewcommand{\geq}{\geqslant}


\begin{document}
	\title{Основы функционального анализа, второй семестр.}
	\author{Тресков Сергей Андреевич}
	\maketitle
	
	\section{Векторные пространства}
	
	\defi \textbf{Векторное пространство} -- это математическая структура, которая формируется набором элементов, называемых векторами, для которых определены операции сложения друг с другом и умножения на число -- скаляр.

	Некоторые примеры векторых пространств:
		\begin{itemize}
			\item Пространства вещественных и комплексных числе $\mathbb{R}^n$ и $\mathbb{C}^n$.
			\item Пространство непрерывных функций $C[a,b]$.
			\item Пространство интегрируемых функций $\mathbb{L}_1$, а так же $\mathbb{L}_p$ -- пространство функций $f$ таких, что $p$-я их степень $|f|^p$ интегрируема.
			\item Пространства функций медленного роста ($\mathbb{S}$) и финитных ($\mathbb{D}$), а также построенные по ним пространства обобщённых функций $\mathbb{S}'$ и $\mathbb{D}'$.
			\item Пространство последовательностей $l_2 : x = (x_1, x_2, ...)$ таких, что $\sum_{i=1} |x_i|^2 < \infty$, а так же пространство ограниченных последовательностей $m = l_\infty : x = (x_1, x_2, ...)$
		\end{itemize}

	В данном курсе преимущественно рассматривается векторное пространство $l_2$. Сразу стоит заметить, что среди перечисленных пространств только $\mathbb{R}^n$ и $\mathbb{C}^n$ имеют конечную размерность, остальные -- бесконечномерны.
	
	\defi Произвольное множество векторов из векторного пространства E называется \textbf{линейно независимым}, если каждое конечное
	подмножество векторов, лежащее в E, тоже линейно независимо.
	
	Для пространства $l_2$ потребуется ввести новое определение предела, используя понятие метрики:
	
	\defi \textbf{Метрическим пространством} называется множество M, в котором определено расстояние $\rho$ между любой парой элементов. Обозначается -- (M, $\rho$), $\rho : M^2 \rightarrow 
	\mathbb{R}$, причем для $\rho$ выполняются следующие условия:
	\begin{enumerate}
		\item $\rho(x,y) \geq 0$~$\&$~$(\rho = 0 \equals x=y)$
		\item $\rho(x,y) = \rho(y,x)$
		\item $\rho(x,z) \leq \rho(x,y) + \rho(y,z)$ (неравенство треугольника)
	\end{enumerate}
	
	Так же, как и в курсе математического анализа, введем определение открытой элементарной окрестности:
	$$B(x, \varepsilon) = \{y \in M | \rho(x,y) < \varepsilon\}$$
	
	Таким образом, возможно определение предела как в терминах расстояний, так и в терминах окрестностей. Здесь приводится первое определение,
	а второе остается в качестве упражнения:
	
	\defi \textbf{Пределом} функции $f : M_1 \rightarrow M_2$ называется $y \in M_2$, такой что $$\forall \varepsilon > 0 ~\exists \delta(\varepsilon) > 0, \forall x \in M_1 ~\&~ x  \neq x_0 : $$
	$$\rho_1(x, x_0) < \delta \Rightarrow \rho_2(f(x), y) > \varepsilon$$
	Обозначение: $lim_{x \rightarrow x_0} f(x) = y$
	
	\defi Назовем подмножество $U \subset M$ \textbf{открытым}, если любая точка в нём содержится вместе с некоторой окрестностью.
	
	\example Рассмотрим дискретное метрическое пространство:
	$$\rho = \{^{1, x \neq y}_{0, x = y}$$
	Любое подмножество, содержащееся в пространстве с такой метрикой является открытым: каждая точка содержит окрестность радиуса 
	$\frac{1}{2}$.
	
	\defi Множество M называется \textbf{замкнутым}, если оно содержит все свои предельные точки.
	
	\defi \textbf{Замыкание} множества -- это объединение множества и всех его предельных точек. Обозначают $\bar{M}$ или $cl ~ M$.
	
	\exc Доказать, что замыкание множества является замкнутым.
	
	\exc Доказать, что дополнение к открытому множеству замкнуто, а к замкнутому открыто.
	
	\defi $M_1 \subset M$, $M_2 \subset M$. Подмножество $M_1$ \textbf{плотно} в $M_2 \equals M_2 \subset \bar{M}_1$
	
	\example Множество рациональных чисел плотно в множестве иррациональных.
	
	\defi Пусть $M_1 \subset M$. $M_1$ \textbf{всюду плотно} $\equals$ $\bar{M}_1 = M$. (То есть для любой точки из M существует последовательность
	из $M_1$, которая сходится к этой точке.)
	
	\defi Множество M называется \textbf{сепарабельным}, если у него найдется счетное, всюду плотное подмножество.
	
	\defi \textbf{Счётное множество} -- множество, все элементы которого можно пронумеровать.

	{\color{gray} Немного о счётности. Самым простым примером счётного множества является множество натуральных чисел $\mathbb{N}$, поскольку нумерация элементов множества как раз и производится натуральными числами. Множество рациональных чисел $\mathbb{Q}$ так же счётно (поскольку его можно представить в виде прямого произведения $\mathbb{N}$ на само себя, а произведение счётных множеств -- счётно), множества $\mathbb{R}$ и $\mathbb{C}$ несчётны.}
	
	\defi Последовательность $\{x_n\}$ называется \textbf{фундаментальной} (или \textbf{последовательностью Коши}), если выполнен \textbf{критерий Коши}:
	$$\forall \varepsilon > 0 ~\exists N = N(\varepsilon),~ \forall n_1, n_2 > N,~ \rho(x_{n_1}, x_{n_2}) < \varepsilon$$

	\defi Метрическое пространство называется \textbf{полным}, если любая содержащаяся в нем фундаментальная последовательность имеет предел.
	
	В любом векторном пространстве существует понятие длины или нормы вектора:
	
	\defi \textbf{Норма вектора} Пусть x - вектор. Назовем нормой $\|...\| : x \rightarrow \mathbb{R_+}$
	\begin{enumerate}
		\item $\|x\| \geq 0 ~(\|x\| = 0 \equals x = 0)$
		\item $\|\alpha x\| = |\alpha| \|x\|$
		\item $\|x + y\| \leq \|x\| + \|y\|$
	\end{enumerate}
	
	\example $\|~\|_{L_1} = \int {|f|}$, $\|~\|_{L_2} ~=~ \sqrt{\int {|f|^2}}$,  $\|~\|_{m} ~= \sup {|m_i|}$
	
	Для нормированного пространства можно легко определить метрику, вводя $\rho(x,y) = \|x-y\|$. При этом будут выполняться все аксиомы,
	определенные для метрики ранее.
	
	Можно рассмотреть два идентичных определения эквивалентности норм:
	
	\defi Две нормы называются \textbf{эквивалентными}, если они порождают один тот же запас открытых множеств.
	
	\defi Две нормы $\rho_1$ и $\rho_2$ на пространстве V называются \textbf{эквивалентными}, если существуют две положительные константы 
	$C_1$ и $C_2$, такие, что для любого $x \in V$ выполняется 
	$$C_1 p(x) \leq q(x) \leq C_2 p(x)$$
	Эквивалентные нормы задают на пространстве одинаковую топологию. \sout{В конечномерном пространстве все нормы эквивалентны.}
	
	\defi \textbf{Банахово} пространство -- полное нормированное пространство.
	
	Значительное внимание в курсе уделено пространствами со скалярным произведением.
	
	\defi Линейное пространство со скалярным произведением. Существует скалярное произведение: $ <...,...> : E^2 \rightarrow C$.
	Для скалярного произведения определены следующие свойства:
	\begin{enumerate} 
		\item $<x,y> = \overline{<y,x>}$
		\item $<\alpha x_1 + \beta x_2, y> ~= ~\alpha <x_1, y> + \beta <x+2, y>$
		\item $<x, x> ~\geq ~0 ~(<x, x> = 0 \equals x = 0)$
	\end{enumerate}
	
	\defi $\|x\| = \sqrt{<x, x>}$ {\color{gray} Пока это <<контрабандное>> утверждение, докажем его позже.}
	
	\exc Доказать тождество параллелограмма: 
	$$\|x+y\|^2 + \|x-y\|^2 = 2 \|x\|^2 + 2 \|y\|^2$$
	Так как пространство $l_2$ - бесконечномерное пространство, для него потребуется доказать неравенство Шварца (оно же неравенство
	Коши-Буняковского):
	$$|<x, y>| \leq \|x\| \cdot \|y\|$$
	Введем две вспомогательные переменные:
	$$\theta = \frac{<x, y>}{|<x, y>|}, t \in \mathbb{R}$$
	
	$$0 \leq \|\bar{\theta} x + t y\|^2 ~=~ <\bar{\theta} x + t y, \bar{\theta} x + t y> ~= $$
	$$\bar{\theta} <x, \bar{\theta} x + t y> ~+~ t <y, \bar{\theta} x + t y> ~= $$
	$$|\theta|^2 \|x\|^2 ~+~ t \bar{\theta} <x, y> ~+~ t \theta <y, x> ~+~ t^2 \|y\|^2 ~= $$
	$$t^2 \|y\|^2 + 2t |<x,y>| + \|x\|^2$$
	Дискриминант должен быть не положительным. Таким образом, получаем:
	$$|<x, y>|^2 \leq \|x\|^2 \|y\|^2$$
	Взяв квадратный корень из обеих частей выражения, получим искомое неравенство.
	
	\defi \textbf{Гильбертовым пространством} называется полное пространство относительно нормы, порожденной скалярным произведением.
	
	\example $\mathbb{R}^n$ - гильбертово пространство.
	
	\defi $x \in l_2$ $\|x\| = (\sum{|x_i|^2})^{1/2}$ 
\end{document}
