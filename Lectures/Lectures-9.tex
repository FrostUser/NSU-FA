	Факты, доказанные на прошлой лекции, послужат основой для доказательства теорем этой лекции.
	Повторно выпишем равносильные утверждения для принадлежности $\lambda$ к регулярным значениям и к спектру.
	
	$$A:H \rightarrow H \text{--- самосопряжённый.}$$
	\begin{align}
		\lambda \in R_A       &\equals \exists C > 0, \forall h \norm{A_{\lambda}h} \geq C\norm{h} \label{eq:regCrit}\\
		\lambda \in \sigma(A) &\equals \exists\{x_n\}, \norm{x_n} = 1, A_{\lambda}x_n \rightarrow 0 \label{eq:specCrit}
	\end{align}
	
	\begin{state}
		Любое $\lambda$, которое не является вещественным, является регулярным значением для самосопряжённого оператора.
		$$\lambda = \alpha + i\beta,\, \beta \neq 0 \Rightarrow \lambda \in R_A$$
	\end{state}
	\begin{proof}
		Рассмотрим два скалярных произведения:
		$$\scal{A_{\lambda}x}{x} = \scal{Ax}{x} - \lambda\scal{x}{x}$$
		$$\scal{x}{A_{\lambda}x} = \scal{x}{Ax} - \scal{x}{\lambda x} =
		\underbrace{\scal{Ax}{x}}_{\mathclap{\text{Из самосопряжённости A}}} - \bar\lambda\scal{x}{x}$$
		
		Вычитая одно уравнение из другого, получим:
		$$-2i\beta\norm{x}^2 = \scal{A_{\lambda}x}{x} - \scal{x}{A_{\lambda}x}$$
		
		Возьмём модуль от обеих частей уравнения. Используя факт, что модуль разности не превосходит суммы модулей и неравенство Шварца, 
		прийдём к неравенству, доказывающему теорему:
		$$2\beta\norm{x}^2 = |\!\scal{A_{\lambda}x}{x} - \scal{x}{A_{\lambda}x}\!| \leq |\!\scal{A_{\lambda}x}{x}\!| + |\!\scal{x}
		{A_{\lambda}x}\!| \leq 2\norm{A_{\lambda}x}\cdot\norm{x}$$
		
		\todo{Быть может, лучше говорить <<критерий регулярности>>?}
		Теперь, то, что $\lambda \in R_A$ очевидно: нужно в равносильном утверждении (\ref{eq:regCrit}) взять $C = \beta$. \\
	\end{proof}
	
	В частности, это утверждение означает, что для самосопряжённого оператора $A$, $\sigma(A)\subset\mathbb{R}$. Тогда имеет смысл
	следующая теорема: (можем определить верхнюю и нижнюю грани)
	
	\begin{theorem}
		\todo{Здесь должны быть ссылки на номера определений про m и M}
		Для самосопряжённого оператора $A$ выполнено $\sigma(A) \subset [m;M]$, где $m$ и $M$ были приведены в прошлой лекции.
	\end{theorem}
	\begin{proof}
		\todo{Аналогично ли?}
		Докажем ограниченность справа. Слева производится аналогично.	
	
		Положим $\lambda = M + d$, где $d > 0$.
		$$\scal{A_{\lambda}x}{x} = \scal{Ax}{x} - \lambda\norm{x}^2 \leq M \norm{x}^2 - \lambda\norm{x}^2 = -d\norm{x}^2$$
		
		Из свойств нормы и условий теоремы ясно, что $-d\norm{x}^2$ --- отрицательное число. В таком случае, при взятии модуля
		от обеих частей неравенства, знак неравенства изменится на противоположный.

		$$\mod{\scal{A_{\lambda}x}{x}} \geq d\cdot\norm{x}^2$$
		
		Отсюда, зная $\norm{A_{\lambda}x}\cdot\norm{x} \geq \mod{\scal{A_{\lambda}x}{x}}$, получаем неравенство
		
		$$\norm{A_{\lambda}x} \geq d\cdot\norm{x}$$
		
		Значит, $\forall d > 0:\: \lambda \in R_A$, в силу (\ref{eq:regCrit}), если выбрать $C~=~d$.
	\end{proof}
	
	Сделаем некоторые важные замечания относительно спектра самосопряжённого оператора. Пусть $A$ --- самосопряжённый. Из свойств
	самосопряжённого оператора, $(-A)$ также является самосопряжённым. Из определения $A_{\lambda} \overset{df}{=} A - \lambda I$ 
	очевидно, что
	$$\lambda \in \sigma(A) \equals (-\lambda) \in \sigma(-A)$$
	Следовательно $\sigma(-A) \subset [-M, -m]$
	
	Рассмотрим спектр оператора $A_{\mu} = A - \mu I$, $\mu \in \mathbb{R}$. Тогда $A = A^* \Rightarrow A_{\mu} = A_{\mu}^*$ 
	(так как $-\mu I$ тоже самосопряжённый для $\mu\in\mathbb{R}$) и видно, что $\sigma(A_{\mu}) \subset[m-\mu,M-\mu]$. Значит,
	сдвигом можно выбрать новые $m'$, $M'$, такие, что $0\leq m' \leq M'$.
	
	Благодаря этим выкладкам, можно воспользоваться теоремой о норме самосопряжённого оператора.
	
	\begin{theorem}
		(О норме самосопряжённого оператора)
		Если $A$ --- самосопряжённый, $0 \leq m \leq M$ (Для любого самосопряжённого оператора можно добиться выполения этого свойства),
		$M \in \sigma(A)$, то $\norm{A} = M$.
	\end{theorem}
	\begin{proof}
		Из условий теоремы и из равносильного утверждения (\ref{eq:specCrit}) следует, что
		$$\exists \{x_n\},\, \norm{x_n} = 1,\, \scal{Ax_n}{x_n} \rightarrow M$$
		\todo{Разбить выкладки на несколько частей с промежуточными пояснениями.}
		Рассмотрим $\norm{A_Mx_n}^2$:
		\begin{equation*}
		\begin{split}
			\norm{A_Mx_n}^2 = \scal{Ax_n - Mx_n}{Ax_n - Mx_n} = \\
			\norm{Ax_n}^2 + M^2\norm{x_n}^2 - 2M\scal{Ax_n}{x_n} 
			\overset{\mathclap{\text{Так как }\norm{x_n}=1}}{\leq} \\
			\leq 2M(M-\underbrace{\scal{Ax_n}{x_n}}_{\mathclap{\rightarrow 0 \text{ по (\ref{eq:specCrit})}}}) 
			\leq M^2\norm{x_n}^2 = M^2
		\end{split}
		\end{equation*}
	\end{proof}
	
	Развернув доказательство в другую сторону, получим, что $m$ --- тоже точка спектра.
	
	\subsection{Инвариантное пространство линейного оператора.}
	\begin{defi}
		\todo{Должен ли $A$ быть линейным? (Я почти уверен, что нет.}
		Пусть определён оператор $A: H \rightarrow H$. $L \subset H$ называется \textbf{инвариантным подпространством}
		оператора $A$, если 
		$$\forall h \in L,\, Ah \in L$$
	\end{defi}
	
	\begin{state}
		$L$ --- инвариантное подпространство $\Rightarrow$ $\overline L$ тоже инвариантно.
	\end{state}
	\begin{theorem}
		Пусть $A$ --- линейный ограниченный оператор. Тогда, если $L$ --- инвариантное подпространство оператора $A$, то 
		$L^{\perp}$ --- инвариантное подпространство для $A^*$.
	\end{theorem}
	Так как $A^{**} = A$, теорема верна в обе стороны. Также, если $A$ --- самосопряжённый, то как $L$, так и 
	$L^{\perp}$ будут его инвариантными подпространствами ($A^* = A$).
	\begin{proof}
		Положим $x \in L$,$y \in L^{\perp}$.
		$$\scal{x}{y} = 0$$
		Но $L$ --- инвариантное подпространство. Значит, можем записать
		$$
		\begin{CD}
			\scal{Ax}{y} @. = 0 \\
			\parallel \\
			\scal{x}{A^*y}
		\end{CD}
		$$
		Таким образом, $\forall x \in L$,$y \in L^{\perp}: \scal{x}{A^*y} = 0$. Тогда $A^*y \in L^{\perp}$, то есть
		$L^{\perp}$~является инвариантным подпространством $A^*$.
	\end{proof}
	
	Рассмотрим самосопряжённый оператор $A:H\rightarrow H$, где $H$ --- гильбертово пространство. Пусть $L$ --- инвариантное 
	подпространство оператора $A$. Тогда, как уже было замечено, $L^{\perp}$ тоже будет инваринантным подпространством.
	
	\todo{Ссылка!}
	Так как $L$ и $L^{\perp}$ --- замкнутые подпространства $H$, они тоже будут гильбертовыми. Поэтому, ограничив $A$ на $L$ или 
	$L^{\perp}$, операторы $A_L$ и $A_{L^{\perp}}$ останутся ограниченными и самосопряжёнными. Тогда можем 
	рассмотреть следующее утверждение:
	\begin{state}
		$\sigma(A) = \sigma(A_L) \cup \sigma(A_{L^{\perp}})$.
	\end{state}
	\begin{proof}
		Пусть в спектре $A_L$ или $A_{L^{\perp}}$ находится собственное число. Для самосопряжённого оператора верно свойство 
		(\ref{eq:specCrit}), то есть найдётся последовательность $\{x_n\}$, $\norm{x_n} = 1$, такая, что 
		$\norm{A_{\lambda}x_n}~\rightarrow~0$. Тогда, так как $\norm{A_{L\lambda}}\leq\norm{A_{\lambda}}$, будет гарантироваться,
		что 
		$$\sigma(A) \subset \big(\sigma(A_L) \cup \sigma(A_{L^{\perp}})\big)$$
		
		Осталось доказать "$\supset$". Рассмотрим $\lambda \notin \sigma(A_L) \cup \sigma(A_{L^{\perp}})$, то есть
		$$
		\left.
		\begin{aligned}
			&\lambda \in R(A_L) \\
			&\lambda \in R(A_{L^{\perp}})
		\end{aligned}
		\right\}
		\Rightarrow \exists C_1 > 0,\, C_2 > 0:\:
		\left\{
		\begin{aligned}
			&\forall x \in L, &\norm{A_{L\lambda}} &\geq C_1\norm{x} \\
			&\forall y \in L^{\perp}, &\norm{A_{L^{\perp}}} &\geq C_2\norm{y}
		\end{aligned}
		\right.
		$$
		
		Тогда возьмём $h \in H = L \oplus L^{\perp}$, где $h = x+y$. Тогда 
		$$\norm{A_{\lambda}h}^2 = \norm{A_{\lambda}x + A_{\lambda}y}^2 
		\underset{\mathclap{\text{(Из ортогональности)}}}{=} \norm{A_{\lambda}x}^2 + \norm{A_{\lambda}y}^2 \geq
		C_1^2\norm{x}^2 + C_2^2\norm{x}^2$$
		Взяв $C = \min(C_1,\,C_2)$, получится
		\begin{gather*}
			\norm{A_{\lambda}h}^2 \geq C^2\norm{h}^2
		\end{gather*}
		
		А это значит, что $\lambda \notin \sigma(A)$ при $\lambda \notin \sigma(A_L) \cup \sigma(A_{L^{\perp}})$. Тогда
		$$\sigma(A) \supset \big(\sigma(A_L) \cup \sigma(A_{L^{\perp}})\big)$$
		
		Таким образом, получили совпадение этих двух множеств, что и требовалось доказать.
	\end{proof}
	
	\subsection{Компактные операторы}
	
	Для определения компактного множества мы будем использовать понятие секвенциальной компактности:
	\begin{defi}
		$K \subset M$ --- \textbf{компактное множество}, если из любой его последовательности можно выделить подпоследовательность,
		которая
		\begin{itemize}
			\item Сходится.
			\item Лежит в $K$.
		\end{itemize}
	\end{defi}
	
	В курсе математического анализа имела место теорема:
	
	\begin{theorem}
		В конечномерном пространстве компактность $\equals$ замкнутости и ограниченности.
	\end{theorem}
	
	В общем случае, равносильность не достигается.
	
	\begin{theorem}
		Компактность $\Rightarrow$ замкнутость и ограниченность.
	\end{theorem}
	\begin{proof}
		\textbf{Замкнутость.} По определению компакта, получим, что любая последовательность из 
		компактного множества, которая имеет предел, лежит в нём. Значит, это множество содержит все свои предельные точки, то есть 
		является замкнутым.
		
		\textbf{Ограниченность.} Пусть найдётся компактное, но не ограниченное множество. Ограниченность означает, что всё множество
		лежит внутри некоторого шара. Следовательно, какого бы радиуса мы не брали шар, всегда найдутся точки вне его. Тогда
		найдётся последовательность, не являющаяся фундаментальной. Следовательно, из такой последовательности не получится выделить
		сходящуюся подпоследовательность --- противоречие с компактностью. Таким образом, компактное множество обязано быть ограниченным.
	\end{proof}
	
	\example Замкнутое и ограниченное, но не компактное множество: \\
	Последовательность $e_n$ (ортонормированных векторов) в единичном шаре в пространстве $l_2$. Такое множество ограниченно, 
	замкнуто, но не является компактным.
	
	\begin{defi}
		Множество $K$ называется \textbf{предкомпактным}, если $\overline{K}$ --- компактно.
	\end{defi}
	
	\example В $\mathbb{R}^n$ любое ограниченное множество предкомпактно.
	
	\example Любой компакт предкомпактен.
	
	\begin{note}
		Предкомпактное множество ограниченно.
	\end{note}
	
	\begin{defi}
		\todo{Точно ли $F$ --- полное?}
		Линейный оператор $A:E \rightarrow F$ (где $F$ --- полное) называется \textbf{компактным} или 
		\textbf{вполне непрерывным}, если он переводит любое ограниченное множество в предкомпактное.
	\end{defi}
	
	Удобно также иметь эквивалентное определение: из образа любой ограниченной последовательности в $E$ можно выбрать 
	фундаментальную (или сходящуюся) подпоследовательность.
	
	\subsubsection{Свойства компактного оператора}
	\begin{enumerate}
		\item $A$ --- компактный оператор $\Rightarrow$ $A$ --- ограниченный.		
		\item $A, B$ --- компактные операторы $\Rightarrow$ $\alpha A + \beta B$ --- компактный оператор.		
		\item Если $A$ --- ограниченный линейный оператор и множество прибытия конечномерно, то $A$ --- компактный.		
		\item Пусть $A$ --- компактный оператор, $B$ и $C$ --- ограниченные операторы, тогда: \label{compCompos}
			\begin{itemize}
				\item $AB$ --- компактный оператор.
				\item $CA$ --- компактный оператор.
			\end{itemize}
		
			\todo{Ссылка на соответствующее свойство}
			Если оператор $C$ ограничен, то он непрерывен, то есть переводит сходящуюся последовательность 
			в сходящуюся. Компактность оператора $A$ означает, что из образа любой ограниченной 
			последовательности можно выделить сходящуюся, как было определено выше.
		
			Тогда компактность $AB$ и $CA$ доказывается следующими утверждениями:
			\begin{itemize}
				\item $B$ переводит ограниченную последовательность в ограниченную.
				\item $C$ переводит сходящуюся последовательность в сходящуюся.
			\end{itemize}		
		\item Тождественный оператор $I:E\rightarrow E$ компактен, если $E$ --- конечномерно.
		\item Если $A:E \rightarrow E$ --- компактный оператор, $dim(E) = \infty$, то $A^{-1}$ не может быть ограниченным. \\

			$A A^{-1} = I$, но для бесконечномерного $E$, $I$ не является компактным оператором. Значит, 
			из свойства \ref{compCompos} $A^{-1}$ не может быть ограниченным.
		\item $A_n$ --- компактный оператор, тогда, если $A_n:H_1 \rightarrow H_2$, $\norm{A_n - A} \rightarrow~0$, то 
		оператор $A$ --- компактный.
		
		Пусть $\{x_n\}$ --- ограниченная последовательность. Нужно выделить из неё подпоследовательность, которую оператор $A$ 
		переведёт в фундаментальную. Выпишем эту подпоследовательность:
		$$x_1^1,\,x_2^1,\,x_3^1,\dots$$
		Эта последовательность тоже будет ограниченной. Из последовательности $\{x^1_n\}$ можно выделить подпоследовательность 
		$\{x^2_n\}$ для $A_2$. Для $A_k$ получится последовательность $\{x^k_n\}$. Покажем, что оператор $A$ переведёт 
		последовательность
		$$x^1_1,\,x^2_2,\,\dots,\,x^k_k,\dots$$
		в фундаментальную:
		\begin{align*}
			\norm{Ax_n^n - Ax_m^m} = \norm{Ax_n^n - A_kx^n_n + A_kx_n^n + A_kx_n^n - A_kx_m^m + A_kx_m^m - Ax_m^m} \leq \\
			\begin{aligned}
				&\leq \norm{(A-A_k)x_n^n} &+ &\norm{A_k(x_n^n - x_m^m)} &+ &\norm{(A_k - A)x_m^m} \leq \\
				&\leq \norm{A-A_k}\cdot\norm{x_n^n} &+ &\norm{A_k(x_n^n - x_m^m)} &+ &\norm{A_k - A} \cdot\norm{x_m^m}
			\end{aligned}
		\end{align*}
		
		Выберем $k$ таким образом, что первое и третье слагаемые в сумме были меньше $\frac{\varepsilon}{2}$. Тогда 
		$$\dots < \frac{\varepsilon}{2} + \norm{A_k(x_n^n - x_m^m)}$$
		$A_k x_n^n$ и $A_k x_m^m$ сходятся к одному и тому же пределу, значит при достаточно больших $n$ и $m$ можем выбрать
		$\norm{A_k(x_n^n - x_m^m)} < \frac{\varepsilon}{2}$. Тогда
		$$\norm{Ax_n^n - Ax_m^m} < \varepsilon$$
		Это означает, что из образа любой ограниченной последовательности (для оператора $A$) можно выделить фундаментальную.
		
		Таким образом, последовательность компактных операторов имеет своим пределом компактный оператор.
	\end{enumerate}