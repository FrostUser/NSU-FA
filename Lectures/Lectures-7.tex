	% В конспекте упущено упоминание того, что было в прошлой лекции.

	%\subsection{Спектр и регулярные значения линейного оператора}
	
	На прошлой лекции был рассмотрен оператор $(A - \lambda I)$, и классификация значений $\lambda$, определяющаяся свойствами этого 
	оператора. Для удобства, в дальнейшем будем использовать обозначение $A_{\lambda}$.
	
	\begin{state}
		Множество регулярных значений $r(A)$ оператора $A$ является открытым.
	\end{state}
	\begin{proof}
		Ранее было доказано, что оператор $(A + \Delta)$ обратим, если $\norm{\Delta} \leq \frac{1}{\norm{A^{-1}}}$ 
		и $A$ является обратимым оператором. Тогда, если $\lambda \in r(A)$, то оператор
		$$ (A - \lambda I) - \mu I $$
		является обратимым для малых $\mu$. Этим доказывается открытость множества $r(A)$.
	\end{proof}
	С другой стороны, для любого линейного оператора верно
	$$
		\overbrace{r(A)}^{ \mathclap{\text{Регулярные значения $A$}} } \cup 
		\underbrace{\sigma_1 (A) \cup \sigma_2 (A) \cup \sigma_3 (A)}_{\text{Спектр $A$}} = 
		r(A) \cup \sigma(A) = \mathbb{C}
	$$
	Данное равенство означает, что, в силу открытости $r(A)$, \textbf{спектр линейного оператора --- закрытое множество}.
	
	\subsection*
	{
		Сопряжённые гильбертовы пространства. \\
		Основная\footnote{В рамках нашего курса.} теорема гильбертова пространства.
	}
	
	\begin{defi}
		Пусть $E$ --- банахово пространство. Тогда $E'$ будем называть 
		\textbf{пространством непрерывных линейных функционалов из $E$ в $\mathbb{C}$} 
		или \textbf{пространством, сопряжённым к $E$}.
	\end{defi}
	
	Данное определение очень похоже на определение обобщённых функций, введённых в предыдущем курсе функционального анализа. 
	Единственным отличием является то, что теперь функционалы задаются в нормированном пространстве, благодаря в отличие от
	обобщённых функций, где топология задавалась исключительно понятием сходимости
	\footnote
	{
		Строго говоря, топология пространства обобщённых функций также может определяться множеством 
		полунорм --- норм, которые могут равняться нулю на ненулевых элементах. Но эта информация выходит
		за рамки данного курса.
	}
	.
	
	Как и для линейных операторов, линейный функционал непрерывен тогда и только тогда, когда он ограничен. Единственное отличие
	между ними заключается в том, что линейные функционалы обладают фиксированным множеством прибытия.
	
	Преимущественно будем использовать $H'$ --- пространство, сопряжённое гильбертову.
	
	Рассмотрим линейный функционал $l: H \rightarrow \mathbb{C}$. Для него определено понятие ядра:
	$$\Ker(l) = \{ h \in H | l \angular{h} = 0\}$$
	В общем случае непрерывного линейного функционала, ядро обладает следующими свойствами:
	\begin{itemize}
		\item $\Ker$ --- не пустое множество. \\
		(В силу линейности в нём обязательно лежит $h = 0$)
		\item $\Ker$ --- линейное подпространство. \\
		(В силу линейности функционала)
		\item $\Ker$ --- замкнутое подпространство. \\
		(Пусть $x_n \in \Ker(l)$, $x_n \rightarrow x_0 \Rightarrow l\angular{x_n} \rightarrow l\angular{x_0}$)
	\end{itemize}
	
	\exc Доказать, что если ядро линейного функционала $f$ замкнуто, то $f$ ограничен.
	
	% TODO: Разобрать, как поставить, чтобы была не теорема 4.1, а теорема Рисса.
	\begin{theorem} 
		Пусть $l H \rightarrow \mathbb{C}$ - линейный ограниченнный функционал, тогда
		\begin{enumerate}
			\item $\existsonly h_l \in H$, такой, что 
			$$\forall h \in H, l \angular{h} = \scal{h}{h_l}$$
			\item $\forall g \in H$, формула (markme!) определяет линейный непрерывный функционал
			$$f\angular{h} \rightarrow \mathbb{C}$$
			$$f\angular{h} = \scal{h}{g}$$
		\end{enumerate}
	\end{theorem}
	\begin{proof}
		Докажем по порядку приведённые пункты теоремы:
		\begin{enumerate}
			\item Обозначим $G = \Ker(l)$. В таком случае, возможны два варианта:
			\begin{enumerate}
				\item Ядро совпадает с гильбертовым пространством: $(G = H)$. \\
				Данное равенство будет означать, что любой вектор пространства обращается функционалом в ноль.
				Следовательно, $h_l = 0$ будет единственным подходящим решением.
				\item Ядро не совпадает с гильбертовым пространством: $G \neq H$. \\
				Так как $G$ --- замкнутое подпространство, то его ортогональное дополнение $G^{\perp} \neq \varnothing$
				
				Тогда можно взять вектор $h_0 \in G^{\perp}, h_0 \neq 0$. Взяв произвольный $h \in H$, рассмотрим вектор
				\begin{equation} \label{eq:RissVector}
					(l\angular{h}) h_0 - (l\angular{h_0}) h
				\end{equation}
				Несложно показать, что вектор \eqref{eq:RissVector} лежит в G. Действительно,
				$$
					l\angular{(l\angular{h}) h_0 - (l\angular{h_0}) h} 
					= l\angular{h}l\angular{h_0} - l\angular{h_0}l\angular{h} = 0
				$$
				Теперь рассмотрим скалярное произведение векторов \eqref{eq:RissVector} и $h_0$:
				$$
					(l\angular{h}) \cdot \norm{h_0}^2 - (l\angular{h_0}) \cdot \scal{h}{h_0} = 0
				$$
				Откуда в результате нехитрых преобразований получается 
				\begin{equation} \label{eq:hlvector}
					h_l = \frac{\overline{l\angular{h_0}}}{\norm{h_0}^2} \cdot h_0
				\end{equation}
			\end{enumerate}
			Докажем единственность полученного $h_l$. Предположим, что это не так и существуют два вектора $h'$ и $h''$, таких, что 
			$$l\angular{h} = \scal{h}{h'} = \scal{h}{h''}$$
			Тогда будет верно
			\begin{eqnarray*}
				\scal{h}{h'} = \scal{h}{h''} \\
				\scal{h}{(h' - h'')} = 0
			\end{eqnarray*}
			Так как данное равенство верно для любых $h$, возьмём $h = h' - h''$. Получим $\norm{h'-h''}^2 = 0$, откуда следует
			$h' = h''$, что и требовалось доказать.
			
			{\footnotesize
				Единственность $h_l$ позволяет судить о размерности $G^{\perp}$.
				Так как \eqref{eq:hlvector} выполнено для всех $h_0 \in G^{\perp}$, вектор $h_l$ параллелен всем $h_0$, 
				что возможно лишь при $\dim{G^{\perp}} = 1$ ($\dim{G^{\perp}} \neq 0$ по предположению доказательства).
			}
			
			\item Рассмотрим функцию $f$, такую, что $f(h) = \scal{h}{g}$. В силу свойств скалярного произведения, 
			$f(h)$ --- линейный функционал, который, в силу неравенства Шварца, будет ограниченным:
			\begin{equation} \label{eq:BoundedProof}
				|f(h)| \leq \norm{h} \cdot \norm{g} \Rightarrow \norm{f} \leq \norm{g}
			\end{equation}
			Подставим $h = g$ в функцию $f(h)$. Получим
			$$|f(g)| = \norm{g}^2$$
			Так как норма оператора, определяется как $\sup \frac{f(h)}{\norm{h}}$, то 
			$$\norm{f} \geq \norm{g}$$
			Рассматривая данное неравенство вместе с первым неравенством из \eqref{eq:BoundedProof}, получаем 
			$$ \norm{f} = \norm{g} $$
			
			Или, в обозначениях предыдущего пункта, $\norm{h_l} = \norm{l}$.
			
			{\footnotesize
				При этом, из $\dim{G^{\perp}} \leq 1$, получим $f(h) = C(h) \cdot \norm{h}^2$, где $C(h)$ --- коэффициент из 
				\eqref{eq:hlvector}. В результате, линейный функционал для любого вектора ограничен как 
				$|f(h)| \leq C \cdot \norm{h^2}$
			}
		\end{enumerate}
	\end{proof}
	
	\begin{note}
		Теорема Рисса устанавливает биекцию $H' \leftrightarrow H$.
	\end{note}
	\begin{proof}
		Рассмотрим формулу $l_1\angular{h} = \scal{h}{h_{l_1}}, h \in H$. Для неё верны следующие
		свойства:
		\begin{itemize}
			\item Оператору $l_1 + l_2$ соответствует вектор $h_{l_1+l_2}$.
			\item Для $\alpha l_1, \alpha \in \mathbb{C}$ $h_{\alpha l_1} = \bar{\alpha} h_l$
		\end{itemize}
	\end{proof}
	
	Введя второе сопряжённое пространство $H''$, увидим следующую связь:
	$$H'' \leftrightarrow H' \leftrightarrow H$$
	При этом, здесь $\leftrightarrow$ обозначает <<гильбертов>> изоморфизм, когда \\
	$h_{\alpha l} = \bar{\alpha} h_l$. % Кривовато выводит, надо будет подумать над правильной вёрсткой.
	Видим, что $H$ и $H''$ связаны обычным изоморфизмом, то есть $h_{\alpha l} = \alpha h_l$.
	% Гильбертов также называется сопряженно линейным, обычный --- канонический.
	
	\subsection{Бра- и кет-векторы.}
	Теорема Рисса отождествляет линейный ограниченный функционал с линейными вектороами гильбертова пространства.
	Но мы привыкли к префиксной записи (а здесь $h_l$, заменяющий функционал, пишется справа), поэтому у физиков
	скалярное произведение линейно по второму аргументу. На текущий и следующий разделы мы тоже <<перейдём в эту веру>>.
	
	$$\scal{f}{g} \overset{df}{=} \pscal{g}{f}$$
	Которое будем называть физическим скалярным произведением, линейным по $g$. Но мы пойдём дальше --- представив
	$\pscal{g}{f} = \bra{g} \ket{f}$, разорвём скалярное произведение. Таким образом, приходим к формализму, 
	созданному Дираком: \\
	\begin{tabular}{l c r}
		$\bra{g}$ & --- & бра-вектор \\
		$\ket{f}$ & --- & кет-вектор \\
	\end{tabular}
	(от английского \textit{bracket} --- скобки)
	
	Авторство данного формализма принадлежит Дираку, который ввёл её для описания квантовых состояний.
	
	Пусть $e_n$ --- гильбертов базис, тогда каждый вектор с точностью совпадает с суммой своего ряда Фурье:
	$$ \forall h,\: h = \sum \alpha_n e_n,\: \alpha_n = \scal{h}{e_n} $$
	Для гильбертова базиса, запишем сначала кет-вектор, потом бра-вектор:
	$$ \ket{e_n} \bra{e_n} $$
	$$ \sum \ket{e_n} \pscal{e_n}{h_n} = \sum \ket{e_n} \alpha_n = h $$
	Так как подобное преобразование переводит $h$ в $h$, результат очевиден:
	$$ \sum \ket{e_n} \bra{e_n} = I_H $$
	
	\subsection{Операторная функция Грина}
	
	Рассмотрим линейный оператор $A : H \rightarrow H$ с собственными числами $\lambda_i$ и собственными векторами $e_i$,
	$$A e_i = \lambda_i e_i$$
	Причём $\{e_i\}$ --- гильбертов базис в $H$.
	
	Теперь рассмотрим $\lambda \neq \lambda_i$ и решим уравнение
	$$ Ax - \lambda x = y $$
	Кет-вектор $\ket{x}$ может быть записан как сумма ряда Фурье:
	$$ \ket{x} = \sum x_n \ket{e_n} $$
	Рассмотрим $A \ket{x} = \sum \lambda_n x_b \ket{e_n}$. Вычтем $\lambda \ket{x}$ из каждой стороны:
	\begin{equation} \label{ketEq}
		A\ket{x} - \lambda\ket{x} = \sum (\lambda_n - \lambda) \cdot x_ \ket{e_n} = \ket{y}
	\end{equation}
	
	Домножим \eqref{ketEq} на бра-вектор $\bra{e_m}$ слева. Тогда, так как $\{e_i\}$ --- гильбертов базис,
	то $(\lambda_n - \lambda) x_n = \pscal{e_m}{y}$.
	$$ x_m = \frac{1}{\lambda_m - \lambda} \cdot \scal{e_m}{y} $$
	Умножая полученное выражение на $\ket{e_m}$, получаем
	$$ \sum_n x_n \ket{e_m} = \sum_m \frac{\ket{e_m}}{\lambda_m - \lambda} \pscal{e_m}{y} $$
	Таким образом, получили выражение исходного вектора через результат оператора:
	$$ \ket{x} = \sum_m \frac{\ket{e_m} \bra{e_m}}{\lambda_m - \lambda} \ket{y} $$
	\begin{equation} \label{greenFunction}
		A^{-1}_{\lambda} = \sum_m \frac{\ket{e_m} \bra{e_m}}{\lambda_m - \lambda}
	\end{equation}
	Равенство \eqref{greenFunction} определяет резоленту и называется \textbf{операторной функцией Грина}.
	
	Выводя данную функцию, мы поступали как физики: не проверяли законность деления и суммирования. Но главное одно:
	результат получен.
	
	\subsection{Сопряжённые операторы}
	Рассмотрим $A: E \rightarrow F$, пусть $f\in F'$ --- линейный ограниченный функционал.
	В таком случае виден функционал $f\angular{A} \in E'$.
	
	\begin{defi} \label{def:adjointOp}
		Для линейного оператора $A: G \rightarrow F$, оператор \\$A^{*}: F' \rightarrow G'$ 
		называется \textbf{сопряжённым оператором}, если он выполняет отображение $f \mapsto g$,
		где $f \in F'$ и $g \in G'$ --- линейные функционалы, и $\forall h \in G, \: g\angular{h} = f\angular{Ah}$
	\end{defi}
	
	Тогда рассмотрим линейный ограниченный оператор $A: H_1 \rightarrow H_2 $, 
	где $H_1, H_2$ --- гильбертовы пространства. Тогда, по теореме Рисса, можем ввести
	непрерывный линейный функционал 
	$$l_2 \in H_2', \: l_2 = \scal{\:}{h_{l_2}}$$
	В таком случае, действием такого функционала на вектор $h_1 \in H_1$ будет
	$$l_2 \angular{h_1} = \scal{A h_1}{h_{l_2}}$$
	Поставим вектору $h_{l_2} \in H_2$ вектор $h_{l_1} \in H_1$, то есть
	$$ l_2 \angular{h_1} = \scal{h_1}{h_{l_1}} = \scal{h_1}{A^{*} h_{l_2}}$$
	
	По определению сопряжённого оператора, для любых $h_1 \in H_1$, $h_2 \in H_2$, будет выполняться равевнство
	\begin{equation} \label{eq:adjointOp}
		\scal{A h_1}{h_2} = \scal{h_1}{A^{*} h_2}
	\end{equation}
	
	{ \color{gray}
		Если на экзамене про сопряжённый оператор будет сказано только \eqref{eq:adjointOp}, полным <<криминалом>> это не будет.
		Тем не менее, для ответа на отличную оценку желательно сказать определение \ref{def:adjointOp}.
	}
	%%% То, что я не знаю куда запихать:
	% Гильбертов базис, в данном курсе является ортонормированной системой 
	% векоторов, которая полна.
	% В гильбертовом базисе разложение вектора по базису --- сумма ряда, в силу бесконечномерности.
	% Гильбертов базис не во всех книгах является ортонормированным.
