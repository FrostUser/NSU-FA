\documentclass[12pt]{article}

\usepackage[utf8]{inputenc}
\usepackage[russian]{babel}

\usepackage{amssymb}
\usepackage{amsmath}
\usepackage{amscd}
\usepackage{amsthm}
\usepackage{xcolor}

\usepackage{indentfirst}

%\usepackage{marginnote} % this is used for notes on the right margin --- \marginnote{\footnotesize txt}

\usepackage{mathtools} % for mathclap command

%\usepackage[normalem]{ulem} % for crossing text out - \sout

% Redefining \def is impossible. I tried, but it is impossible.
%\let\def_prev\def

%%%%%%%%%%%%%%%%%%%%%%%%%%%%%%%%%%%%%%%%%%%%%%%
%           MATH OPERATORS SPACING            %
%%%%%%%%%%%%%%%%%%%%%%%%%%%%%%%%%%%%%%%%%%%%%%%

\let\existstemp\exists
\let\foralltemp\forall
\renewcommand{\exists}{\: \existstemp \:}
\newcommand{\existsonly}{\: \existstemp ! \:}
\renewcommand{\forall}{\: \foralltemp \:}

%%%%%%%%%%%%%%%%%%%%%%%%%%%%%%%%%%%%%%%%%%%%%%%
%            COMMAND SHORTHANDS               %
%%%%%%%%%%%%%%%%%%%%%%%%%%%%%%%%%%%%%%%%%%%%%%%

\newcommand{\example}{{\itshape Пример. }}
\newcommand{\equals}{\Leftrightarrow}
\newcommand{\exc}{{\bfseries Упражнение. }}
\newcommand{\norm}[1]{\left\| #1 \right\|}
\newcommand{\scal}[2]{\left\langle #1, #2 \right\rangle}
\newcommand{\angular}[1]{\langle #1 \rangle}

\newcommand{\Sum}[2]{\underset{#1}{\overset{#2}{\sum}}}
\newcommand{\Int}[2]{\underset{#1}{\overset{#2}{\int}}}
\newcommand{\Ker}{\text{Ker}}

% Physicists' variant of dot product
\newcommand{\pscal}[2]{\, \langle #1 | #2 \rangle \,}
\newcommand{\bra}[1]{\, \langle #1 |}
\newcommand{\ket}[1]{| #1 \rangle \,}

\renewcommand{\leq}{\leqslant}
\renewcommand{\geq}{\geqslant}

%%%%%%%%%%%%%%%%%%%%%%%%%%%%%%%%%%%%%%%%%%%%%%%
%         THEOREM DEFINITION LINES            %
%%%%%%%%%%%%%%%%%%%%%%%%%%%%%%%%%%%%%%%%%%%%%%%

\newtheorem{lem}{Лемма}[section]
\newtheorem{note}{Замечание}[section]
\newtheorem{defi}{Определение}[section]
\newtheorem{theorem}{Теорема}[section]
\newtheorem{state}{Утверждение}[section] % statement

%%%%%%%%%%%%%%%%%%%%%%%%%%%%%%%%%%%%%%%%%%%%%%%
%             GRAPHICS INCLUSION              %
%%%%%%%%%%%%%%%%%%%%%%%%%%%%%%%%%%%%%%%%%%%%%%%

\usepackage{graphicx}

\graphicspath{{./Graphics/}}

%%%%%%%%%%%%%%%%%%%%%%%%%%%%%%%%%%%%%%%%%%%%%%%
%               DRAFT TEMPLATES               %
%%%%%%%%%%%%%%%%%%%%%%%%%%%%%%%%%%%%%%%%%%%%%%%

%\usepackage{marginnotes}
\newcommand{\todo}[1]{\marginpar{\color{red} \tiny #1}}

\begin{document}
	На прошлой лекции доказали, что тождественный оператор может быть компактен, только если область его прибытия конечномерна.
	
	Подобным способом докажем ещё одно утверждение.
	
	Пусть $A: H \rightarrow H$ --- компактный оператор. Введём $L$ --- совокупность собственных векторов 
	оператора $A$, соответствующих собственным значениям $\mod{\lambda} \geq \alpha > 0$ 
	($L$ --- ортонормированная система).
	\begin{state} \label{st:limLambda}	
		$L$ может содержать только конечное число {\color{gray}линейно независимых} векторов.		
		{\color{gray} То есть размерность $L$ конечна.}
	\end{state}
	\begin{proof}
		Пусть таких векторов найдётся бесконечно много.
		\begin{gather*}
		 	e_1      ,\, \dots,\, e_n      ,\, \dots \\
			\lambda_1,\, \dots,\, \lambda_n,\, \dots
		\end{gather*}
		Но эта последовательность ограничена, то есть можно выделить подпоследовательность, которую $A$ переведёт в фундаментальную.
		
		\begin{equation} \label{eq:limDimL}
			\norm{Ae_i - Ae_j}^2 = \norm{\lambda_ie_i - \lambda_je_j}^2 = \norm{\lambda_i}^2 + \norm{\lambda_j}^2 \geq 2\alpha^2
		\end{equation}
		
		Расстояние между всеми элементами образа последовательности ограничено снизу. Значит, фундаментальную последовательность
		выделить не получится. \color{gray}(Действительно, при $\varepsilon < \sqrt{2}\alpha$, выражение (\ref{eq:limDimL}) будет 
		противоречить определению фундаментальной последовательности)
	\end{proof}
	
	\begin{theorem}
		Пусть $A: H \rightarrow H$ --- компактный самосопряжённый оператор. Тогда имеет место следующее утверждение:
		$$\lambda \neq 0 \& \lambda \in \sigma(A) \Rightarrow \lambda \in \sigma_d $$
	\end{theorem}
	\begin{proof}
		Пусть $\lambda \neq 0,\, \lambda \in \sigma(A)$. Тогда, по свойству (\label{eq:specCrit}) найдётся последовательность 
		векторов $\{x_n\}$, такая, что:
		
		$$\norm{x_n} = 1,\, A_{\lambda}x_n \rightarrow 0$$
		
		Используем компактность оператора $A$. Так как $\{x_n\}$ --- ограниченная последовательность ($\norm{x_n} = 1$), то
		мы можем выбрать последовательность $\{x_{n_k}\}$, которую $A$ переведёт в фундаментальную. 
		Положим $Ax_{n_k}\rightarrow y_0$. При этом, так как $\{x_{n_k}\}$ --- подпоследовательность $\{x_{n}\}$, 
		сохранится свойство $Ax_{n_k} \rightarrow 0$.
		$$
		\begin{CD}
			A_{\lambda}x_{n_k} @.=@. Ax_{n_k} @.-@. \lambda x_{n_k} \\
			@VVV @. @VVV @. @VVV \\
			0 @.@. y_0 @.@. ?
		\end{CD}
		$$
		
		Отсюда очевидно следует, что $\lambda x_{n_k} \rightarrow y_0$. Тогда, для второго предела получаем (при $\lambda \neq 0$)
		$$A\dfrac{y_0}{\lambda} = y_0 \Rightarrow Ay_0 = \lambda y_0$$
		
		То есть $y_0$ --- собственный вектор и тогда $\lambda \in \sigma_d(A)$.
	\end{proof}
	
	Пусть $A$ --- компактный, самосопряжённый оператор. Рассмотрим точки его спектра, лежащие в выколотой окрестности нуля. 
	По утверждению \ref{st:limLambda}, найдётся только конечное число таких точек. Пронумеруем их следующим образом:
	\todo{Картинка.}
	\begin{gather*}
		0 < \dots \leq \lambda_3 \leq \lambda_2 \leq \lambda_1 \\
		\lambda_{-1} \leq \lambda_{-2} \leq \dots < 0
	\end{gather*}
	Нумеруем таким образом, потому что собственные значения могут стремиться только к нулю.
	
	\todo{Восстановить абзац про приближённое вычисление $\lambda_1$}
	
	Чтобы найти максимальное собственное значение $\lambda_1$, берутся все одномерные подпространства в $L$. 
	Рассматривается ортогональное дополнение к этим подпространствам и из супремумов квадратичной формы на
	этих ортогональных дополнениях берётся точная нижняя грань.
	
	\subsection{Альтернатива Фредгольма}
	
	\subsection{Теорема Гильберта-Шмидта}
	\begin{theorem}
		(Гильберта-Шмидта) (О существовании базиса/диагонализируемости компактного самосопряжённого оператора)
		
		Если оператор $A : H \rightarrow H$ --- компактный и самосопряжённый, то в $H$ существует гильбертов 
		базис, состоящий только из собственных векторов оператора $A$.
	\end{theorem}
	Ранее, когда рассматривался формализм Дирака (бра- и кет-векторы), требовался гильбертов базис собственных
	векторов. Эта теорема позволяет показать, когда такой базис существует.
	\begin{proof}
		\todo{Всё равно расписать.}
		Если $\norm{A} = 0$, доказательство тривиально.
		
		\todo{Выписать пояснение из тетради}
		Положим $\norm{A} \neq 0 \Rightarrow \exists \lambda_1$ --- собственное значение и соответствующий ему 
		собственный вектор $e_{\lambda_1}$, такой, что $\mod{\lambda_1} = \norm{A}$.
		
		Рассмотрим $\{e_{\lambda_1}\} = L_1$ --- инвариантное подпространство, тогда $L_1^{\perp} = H_1$ --- тоже инвариантное 
		подпространство. Оператор $A_2 = A|_{H_2}$ тоже будет компактным и самосопряжённым. Поэтому так же найдётся
		$\lambda_2,\, \norm{A_2} = \mod{\lambda_2} \leq \mod{\lambda_1}$
		
		Далее рассматриваем $L_2 = \{e_{\lambda_2}\} \cup \{e_{\lambda_1}\}$ и $H_3 = L_2^{\perp}$. Продолжая этот 
		процесс, получим такие соотношения между множествами:
		
		\begin{gather*}
			L_1 \subset L_2 \subset L_3 \dots \\
			H \supset H_2 \supset H_3 \dots
		\end{gather*}
		
		Подобный цикл вложений может либо оказаться конечным, либо окажется так, что эта процедура никогда не закончится.
		% Переставлены абзацы про конечную и бесконечную процедуру.
		
		Пусть этот алгоритм никогда не закончится
		\footnote{
			\begin{verse}
				Проклятый, вечный, грузный, ледяной; \\
				Всегда такой же, он все так же длится.
			\end{verse}
		}.
		Тогда найдётся ортогональная последовательность собственных векторов. Соответствующие им $\mod{\lambda_n}$ 
		формируют монотонную последовательность, поэтому $\mod{\lambda_n} \rightarrow 0$. При этом 
		$\norm{A_m} \geq \norm{A_{m+1}}$. В множестве $\overline{L}_n$ собственные вектора $\{e_{\lambda_i}\}$ 
		формируют гильбертов базис, при этом $\overline{L}_n$ --- инвариантное подпространство оператора
		$A$ (Так как $A$ ограничен $\equals$ непрерывен). Введём $L_{\infty}$ --- пересечение замыканий $L_i$. Тогда 
		получаем, что
		$$L_1 \subset L_2 \subset \dots \subset L_n \subset \dots \subset L_{\infty}$$
		Так как $A$ --- самосопряжённый оператор, то $L^{\perp}_{\infty} = H_{\infty}$ --- тоже инвариантное 
		подпространство.
	\end{proof}
\end{document}