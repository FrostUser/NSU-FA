\documentclass[12pt]{article}

\usepackage[utf8]{inputenc}
\usepackage[russian]{babel}
\usepackage[T1]{fontenc}

\usepackage{amssymb}
\usepackage{amsmath}
\usepackage{amscd}
\usepackage{amsthm}
\usepackage{xcolor}

\usepackage{indentfirst}

\usepackage{marginnote} % this is used for notes on the right margin --- \marginnote{\footnotesize txt}

\usepackage{mathtools} % for mathclap command

%\usepackage[normalem]{ulem} % for crossing text out - \sout

%%%%%%%%%%%%%%%%%%%%%%%%%%%%%%%%%%%%%%%%%%%%%%%
%           MATH OPERATORS SPACING            %
%%%%%%%%%%%%%%%%%%%%%%%%%%%%%%%%%%%%%%%%%%%%%%%

\let\existstemp\exists
\let\foralltemp\forall
\renewcommand{\exists}{\: \existstemp \:}
\newcommand{\existsonly}{\: \existstemp ! \:}
\renewcommand{\forall}{\: \foralltemp \:}

%%%%%%%%%%%%%%%%%%%%%%%%%%%%%%%%%%%%%%%%%%%%%%%
%            COMMAND SHORTHANDS               %
%%%%%%%%%%%%%%%%%%%%%%%%%%%%%%%%%%%%%%%%%%%%%%%

\newcommand{\example}{{\itshape Пример. }}
\newcommand{\equals}{\Leftrightarrow}
\newcommand{\exc}{{\bfseries Упражнение. }}
\newcommand{\norm}[1]{\left\| #1 \right\|}
\newcommand{\scal}[2]{\left\langle #1, #2 \right\rangle}
\newcommand{\angular}[1]{\langle #1 \rangle}
\renewcommand{\mod}[1]{\left| #1 \right|}

\newcommand{\supp}{\,\text{supp}\,}

\newcommand{\Sum}[2]{\underset{#1}{\overset{#2}{\sum}}}
\newcommand{\Int}[2]{\underset{#1}{\overset{#2}{\int}}}
\newcommand{\Ker}{\text{Ker}}
\newcommand{\Exists}{\text{\raisebox{-1pt}{\large$\exists$}}}
\renewcommand{\Im}{\text{Im}}
\newcommand{\Lims}[2]{ \underset{#1}{\overset{#2}{\Big|}}  }
\newcommand{\lims}[2]{\Big|_{#1}^{#2}}

% Physicists' variant of dot product
\newcommand{\pscal}[2]{\, \langle #1 | #2 \rangle \,}
\newcommand{\bra}[1]{\, \langle #1 |}
\newcommand{\ket}[1]{| #1 \rangle \,}

\newcommand{\opt}[1]{{\footnotesize #1 \par}}
\newcommand{\off}[1]{{\color{gray}#1}}

%%%%%%%%%%%%%%%%%%%%%%%%%%%%%%%%%%%%%%%%%%%%%%%
%       DOCUMENT STYLE REDEFINITIONS          %
%%%%%%%%%%%%%%%%%%%%%%%%%%%%%%%%%%%%%%%%%%%%%%%

\renewcommand{\leq}{\leqslant}
\renewcommand{\geq}{\geqslant}

% Почему-то не всегда работает.
%\let\reftemp\ref
%\renewcommand{\ref}[1]{(\reftemp{#1})}

%%%%%%%%%%%%%%%%%%%%%%%%%%%%%%%%%%%%%%%%%%%%%%%
%         THEOREM DEFINITION LINES            %
%%%%%%%%%%%%%%%%%%%%%%%%%%%%%%%%%%%%%%%%%%%%%%%

\newtheorem{lem}{Лемма}[section]
\newtheorem{note}{Замечание}[section]
\newtheorem{defi}{Определение}[section]
\newtheorem{theorem}{Теорема}[section]
\newtheorem*{theorem*}{Теорема}
\newtheorem{state}{Утверждение}[section] % statement

%%%%%%%%%%%%%%%%%%%%%%%%%%%%%%%%%%%%%%%%%%%%%%%
%             GRAPHICS INCLUSION              %
%%%%%%%%%%%%%%%%%%%%%%%%%%%%%%%%%%%%%%%%%%%%%%%

\usepackage{graphicx}
\usepackage{subcaption} % for \begin{subtable}
\usepackage{float}

\graphicspath{{./Graphics/}}

%%%%%%%%%%%%%%%%%%%%%%%%%%%%%%%%%%%%%%%%%%%%%%%
%                  COPYRIGHTS                 %
%%%%%%%%%%%%%%%%%%%%%%%%%%%%%%%%%%%%%%%%%%%%%%%

\newcommand{\copyvar}[1]{\par\opt{В.А. Александров, А.А. Егоров. Вариационное исчисление. #1}}

%%%%%%%%%%%%%%%%%%%%%%%%%%%%%%%%%%%%%%%%%%%%%%%
%               DRAFT TEMPLATES               %
%%%%%%%%%%%%%%%%%%%%%%%%%%%%%%%%%%%%%%%%%%%%%%%

%\usepackage{marginnotes}
\newcommand{\todo}[1]{\marginpar{\color{red} \tiny #1}}
\newcommand{\lecture}[1]{\marginpar{\color{blue} \tiny лекция #1}}

\begin{document}
	На прошлой лекции доказали, что тождественный оператор может быть компактен, только если область его прибытия конечномерна.
	
	Подобным способом докажем ещё одно утверждение.
	
	Пусть $A: H \rightarrow H$ --- компактный оператор. Введём $L$ --- совокупность собственных векторов 
	оператора $A$, соответствующих собственным значениям $\mod{\lambda} \geq \alpha > 0$ 
	($L$ --- ортонормированная система).
	\begin{state} \label{st:limLambda}	
		$L$ может содержать только конечное число {\color{gray}линейно независимых} векторов.		
		{\color{gray} То есть размерность $L$ конечна.}
	\end{state}
	\begin{proof}
		Пусть таких векторов найдётся бесконечно много.
		\begin{gather*}
		 	e_1      ,\, \dots,\, e_n      ,\, \dots \\
			\lambda_1,\, \dots,\, \lambda_n,\, \dots
		\end{gather*}
		Но эта последовательность ограничена, то есть можно выделить подпоследовательность, которую $A$ переведёт в фундаментальную.
		
		\begin{equation} \label{eq:limDimL}
			\norm{Ae_i - Ae_j}^2 = \norm{\lambda_ie_i - \lambda_je_j}^2 = \norm{\lambda_i}^2 + \norm{\lambda_j}^2 \geq 2\alpha^2
		\end{equation}
		
		Расстояние между всеми элементами образа последовательности ограничено снизу. Значит, фундаментальную последовательность
		выделить не получится. \color{gray}(Действительно, при $\varepsilon < \sqrt{2}\alpha$, выражение (\ref{eq:limDimL}) будет 
		противоречить определению фундаментальной последовательности)
	\end{proof}
	
	\todo{Используется ли это где-нибудь?}
	Введём множество $N_{\lambda}$, которое определяется следующим образом:
	$$N_{\lambda} = \{0\} \cup \{\text{собственные вектора $A$, соответствующие $\lambda$}\}$$
	
	\begin{note}
		Если $A$ --- компактный и $\lambda \neq 0$, то $\dim N_{\lambda} < \infty$.
	\end{note}
	\begin{note}
		Если $A$ --- компактный и самосопряжённый, то число его собственных векторов конечно.
	\end{note}
	
	\begin{theorem}
		Пусть $A: H \rightarrow H$ --- компактный самосопряжённый оператор. Тогда имеет место следующее утверждение:
		$$\lambda \neq 0 \& \lambda \in \sigma(A) \Rightarrow \lambda \in \sigma_d $$
	\end{theorem}
	\begin{proof}
		Пусть $\lambda \neq 0,\, \lambda \in \sigma(A)$. Тогда, по свойству (\label{eq:specCrit}) найдётся последовательность 
		векторов $\{x_n\}$, такая, что:
		
		$$\norm{x_n} = 1,\, A_{\lambda}x_n \rightarrow 0$$
		
		Используем компактность оператора $A$. Так как $\{x_n\}$ --- ограниченная последовательность ($\norm{x_n} = 1$), то
		мы можем выбрать последовательность $\{x_{n_k}\}$, которую $A$ переведёт в фундаментальную. 
		Положим $Ax_{n_k}\rightarrow y_0$. При этом, так как $\{x_{n_k}\}$ --- подпоследовательность $\{x_{n}\}$, 
		сохранится свойство $Ax_{n_k} \rightarrow 0$.
		$$
		\begin{CD}
			A_{\lambda}x_{n_k} @.=@. Ax_{n_k} @.-@. \lambda x_{n_k} \\
			@VVV @. @VVV @. @VVV \\
			0 @.@. y_0 @.@. ?
		\end{CD}
		$$
		
		Отсюда очевидно следует, что $\lambda x_{n_k} \rightarrow y_0$. Тогда, для второго предела получаем (при $\lambda \neq 0$)
		$$A\dfrac{y_0}{\lambda} = y_0 \Rightarrow Ay_0 = \lambda y_0$$
		
		То есть $y_0$ --- собственный вектор и тогда $\lambda \in \sigma_d(A)$.
	\end{proof}
	
	Пусть $A$ --- компактный, самосопряжённый оператор. Рассмотрим точки его спектра, с выколотой окрестностью нуля. 
	По утверждению \ref{st:limLambda}, найдётся только конечное число таких точек. Если на всём множестве их бесконечно много, 
	то их число будет расти по мере сужения окрестности. Но это число всё время будет конечным. Значит, только ноль может быть
	предельной точкой. Тогда эти точки можно упорядочить:
	\todo{Картинка.}
	\begin{gather*}
		0 < \dots \leq \lambda_3 \leq \lambda_2 \leq \lambda_1 \\
		\lambda_{-1} \leq \lambda_{-2} \leq \dots < 0
	\end{gather*}
	
	\todo{Оставить этот абзац?}
	{\color{gray}Чтобы найти максимальное собственное значение $\lambda_1$, берутся все одномерные подпространства в $L$. 
	Рассматривается ортогональное дополнение к этим подпространствам и из супремумов квадратичной формы на
	этих ортогональных дополнениях берётся точная нижняя грань.}
	
	Как приближённо найти $\lambda_1$? Как было доказано ранее, $\lambda_1 = M$. Процесс поиска сводится к поиску 
	$\underset{x\neq0}{\sup}{\frac{\norm{Ax}}{\norm{x}}}$.
	
	\subsection{Альтернатива Фредгольма}
	
	\begin{theorem}
		(Альтернатива Фредгольма) Рассмотрим компактный оператор $A: E\rightarrow E$, где $E$ --- банахово 
		пространство и следующие уравнения:
		\begin{align}
			Ax - x &= z   \tag{но} \label{eq:fred1} \\
			Ax - x &= 0   \tag{о} \label{eq:fred2} \\
			A^*y - y &= 0 \tag{со} \label{eq:fred3}
		\end{align}
		
		Тогда возможны два варианта:
		\begin{enumerate}
			\item Однородные уравнения \ref{eq:fred2} и \ref{eq:fred3} имеют нулевые решения. \label{prop:fred1}\\
			Тогда \ref{eq:fred1} имеет решение при любой правой части.
			
			\item Однородное уравнение \ref{eq:fred2} имеет ненулевое решение. \label{prop:fred2}\\
			Тогда размерности пространств решений \ref{eq:fred2} и \ref{eq:fred3} совпадают,
			а уравнение \ref{eq:fred1} имеет решение, если его правая часть ортогональна 
			пространству решений \ref{eq:fred3}.
		\end{enumerate}
	\end{theorem}
	Здесь мы приведём только доказательство данной теоремы для гильбертовых пространств, причём только ту её 
	часть, которая не зависит от компактности оператора $A$. (То есть только пункт \ref{prop:fred2})
	\begin{proof}
		Пусть $f$ --- решение уравнения (\ref{eq:fred3}), то есть $A^*f - f = 0$ или, что то же самое, 
		$(A^* - I)f = 0$. Так как $(A^* - I): H \rightarrow H$, можем рассмотреть скалярное произведение:
		
		\begin{align*}
			\forall h \in H \qquad &\scal{h}{(A^*-I)f} = 0 \\
			                       &\scal{(A-I)h}{f} = 0
		\end{align*}
		
		Таким образом, получаем, что образ оператора $(A-I)$, ортогонален $f$. Значит, ортогональность 
		правой части является необходимым условием для существования решения.
		
		Развернув данное доказательство, можно получить, что $f$ --- решение (\ref{eq:fred3}).
	\end{proof}
	
	\subsection{Теорема Гильберта-Шмидта}
	\begin{theorem}
		(Гильберта-Шмидта) (О существовании {\color{gray}собственного} базиса
		/диагонализируемости компактного самосопряжённого оператора)
		
		Если оператор $A : H \rightarrow H$ --- компактный и самосопряжённый, то в $H$ существует гильбертов 
		базис, состоящий только из собственных векторов оператора $A$.
	\end{theorem}
	Ранее, когда рассматривался формализм Дирака (бра- и кет-векторы), требовался гильбертов базис собственных
	векторов. Эта теорема позволяет показать, когда такой базис существует.
	\begin{proof}
		\todo{Всё равно расписать.}
		Если $\norm{A} = 0$, доказательство тривиально.
		
		\todo{Выписать пояснение из тетради}
		Положим $\norm{A} \neq 0 \Rightarrow \exists \lambda_1$ --- собственное значение и соответствующий ему 
		собственный вектор $e_{\lambda_1}$, такой, что $\mod{\lambda_1} = \norm{A}$.
		
		Рассмотрим $\{e_{\lambda_1}\} = L_1$ --- инвариантное подпространство, тогда $L_1^{\perp} = H_1$ --- тоже инвариантное 
		подпространство. Оператор $A_2 = A|_{H_2}$ тоже будет компактным и самосопряжённым. Поэтому так же найдётся
		$\lambda_2,\, \norm{A_2} = \mod{\lambda_2} \leq \mod{\lambda_1}$
		
		Далее рассматриваем $L_2 = \{e_{\lambda_2}\} \cup \{e_{\lambda_1}\}$ и $H_3 = L_2^{\perp}$. Продолжая этот 
		процесс, получим такие соотношения между множествами:
		
		\begin{gather*}
			L_1 \subset L_2 \subset L_3 \dots \\
			H \supset H_2 \supset H_3 \dots
		\end{gather*}
		
		Подобный цикл вложений может либо оказаться конечным, либо окажется так, что эта процедура никогда не закончится.

		\begin{itemize}
		\item Пусть на какой-то итерации данная процедура закончится, то есть $\exists H_n:\: H \supset H_2 \supset \dots \supset H_n$,
			где $L_1 \subset L_2 \dots \subset L_n$. Так как процедура закончилась, то $A|_{H_n}$ --- нулевой оператор (<<разобрали>>
			все $\lambda \neq 0$). Учитывая, что $H_i$ выбирались как ортогональные дополнения, получили $H = L_n \oplus H_n$. 
			Взяв из $L_n$ любой ортонормированный базис, получаем базис во всём гильбертовом пространстве.
		
		\item Пусть этот алгоритм никогда не закончится
			\footnote
			{
				\\
				Проклятый, вечный, грузный, ледяной; \\
				Всегда такой же, он всё так же длится.
			}.
			Тогда найдётся ортогональная последовательность собственных векторов. Соответствующие им $\mod{\lambda_n}$ 
			формируют монотонную последовательность {\color{gray}(мы пронумеровали их в порядке убывания)}, поэтому 
			$\mod{\lambda_n} \rightarrow 0$. При этом 
			$\norm{A_m} \geq \norm{A_{m+1}}$. В множестве $\overline{L}_n$ собственные вектора $\{e_{\lambda_i}\}$ 
			формируют гильбертов базис, при этом $\overline{L}_n$ --- инвариантное подпространство оператора
			$A$ (Так как $A$ ограничен $\equals$ непрерывен). Введём $L_{\infty}$ --- объединение 
			замыканий всех $L_i$. Тогда получаем, что
			$$L_1 \subset L_2 \subset \dots \subset L_n \subset \dots \subset L_{\infty}$$
			Так как $A$ --- самосопряжённый оператор, то $L^{\perp}_{\infty} = H_{\infty}$ --- тоже инвариантное 
			подпространство. Поймём, что $A_{H_{\infty}}$ --- нулевой оператор. Пусть это не так. Тогда существует
			собственный вектор с соответствующим ему собственным значением $\lambda \neq 0$. Но все
			$\lambda_i \neq 0$ уже использованы. Значит, $A_{H_{\infty}}$ --- нулевой оператор.
			
			Тогда $H = L_{\infty} \oplus L^{\perp}_{\infty}$, где $H_{\infty} = L^{\perp}_{\infty}$ состоит только
			из собственных векторов, соответствующих $\lambda = 0$.
		\end{itemize}
	\end{proof}
	
	Как можно было ожидать, оператор Гильберта-Шмидта имеет отношение к теореме Гильберта-Шмидта.
	
	\begin{state}
		Оператор Гильберта-Шмидта компактен.
	\end{state}
	\begin{proof}
		В любом пространстве $\mathbb{L}_2(\mathbb{R}^n)$ найдётся счётный гильбертов базис 
		(тем самым, это сепарабельные пространства).
		
		\vspace{2pt}\hrule\vspace{2pt}
		
		\todo{Вставить материал с семинара + пояснения}
		
		\opt{
		Поясним: мы знаем, что в пространстве $\mathbb{L}_2(\mathbb{R})$ есть гильбертов базис $\{e^{inx}\}$.
		{\color{gray} Здесь нужно ещё добавить пояснений, но у меня они ужасно записаны}
		Тогда, в случае $\mathbb{R}^2$ достаточно рассмотреть функцию $\psi_n(x,y) = \varphi_n(x) \overline{\varphi_n}(y)$.
		Понятно, что такие функции будут ортогональны. Для $\mathbb{R}^3$ можно перенумеровать эти функции каким-нибудь образом
		и снова получить счётный ортонормированный базис.\opt
		}
		
		\vspace{2pt}\hrule\vspace{2pt}
		
		Рассмотрим случай $n=1$, хотя, как сказано выше, теорема выполнена $\forall n$. 
		%\todo{Точно ли это? Или я неправильно понял?}
		%{\color{gray}(С физиков это будет спрашиваться)}
		$$(Af)(x) = \int_{\mathbb{R}} k(x,t) f(t) dt$$
		
		Знаем, что $\norm{A} \leq \norm{k}_{\mathbb{L}_2(\mathbb{R}^2)}$
		
		Возьмём базис в $\mathbb{L}_2$: $\varphi_n(x)\overline{\varphi_m}(t)$.
		
		Тогда функция $k(x,t)$ может быть представлена рядом Фурье:
		$$k(x,t) = \sum_{n,m}^{\infty}\lambda_{n,m}\varphi_n(x)\overline{\varphi}_m(t)$$
		
		Можем записать это как сходимость по норме к сумме $N$ слагаемых ряда Фурье при $N \rightarrow \infty$:
		\begin{equation} \label{eq:partialFourierSum}
			\norm{k(x,t) - \sum_{n,m\leq N}\lambda_{n,m}\varphi_n(x)\overline{\varphi}_m(t)} \rightarrow 0
		\end{equation}
		
		Рассмотрим оператор Гильберта-Шмидта $A_N$ с ядром $k_N$.
		\begin{gather}
			k_N(x,t) = \sum_{m,n \leq N} \lambda_{n,m} \varphi_n(x)\overline{\varphi}_m(t) \\
			(A_Nf)(x) = \int_{\mathbb{R}} k_N(x,t) f(t) dt
		\end{gather}
		
		Образ $A_N$ попадает в $\left\{\sum_1^N \alpha_i \varphi_i \right\} \Rightarrow$ образ оператора $A_N$ 
		конечномерен. Тогда, так как оператор $A_N$ --- ограничен, он компактен по свойству (\ref{compArea}) компактного оператора.
		С другой стороны, из (\ref{eq:partialFourierSum}) $A_N \rightrightarrows A$. Тогда, по свойству (\ref{compLim}), оператор
		$A$ тоже будет компактным.
	\end{proof}
	
	\todo{Неясно, адекватно ли говорить здесь <<в связи>>...}
	Далее некоторая часть нашего курса будет посвящена решению интегральных уравнений. В связи с этим введём несколько определений:
	
	\begin{defi}
		\textbf{Уравнениями Фредгольма} называются следующие уравнения:
		\begin{align*}
			&(Ax)(t) + f(t) = 0 &\qquad &\text{Уравнение Фредгольма I рода} \\
			&(Ax)(t) + f(t) = x(t) &\qquad &\text{Уравнение Фредгольма II рода}
		\end{align*}
	\end{defi}
	
	\todo{Здесь было ещё одно утверждение, которое желательно уточнить.}
	% $t > x,\, k(x,t) = 0$
	Преимущественно мы будем рассматривать ситуацию 
	$$\int_a^b \supp k(x,t) \subset [a,b]^2$$
	Что означает
	$$A_k: \mathbb{L}_2(a,b) \rightarrow \mathbb{L}_2(a,b)$$
	
	Зачастую будет рассматриваться \textbf{уравнение Вольт\'eра} II рода:
	$$\int_a^t k(t,u) x(u) du + f(t) = x(t)$$
	Это уравнение, как будет доказано на следующей лекции, является частным случаем уравения Фредгольма 
	II рода. Это уравнение рассматривается отдельно, так как для него доказаны существование и единственность
	решения.
\end{document}