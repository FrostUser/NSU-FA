\documentclass[12pt]{article}

\usepackage[utf8]{inputenc}
\usepackage[russian]{babel}

\usepackage{amssymb}
\usepackage{amsmath}
\usepackage{amscd}
\usepackage{amsthm}
\usepackage{xcolor}

\usepackage{indentfirst}

%\usepackage{marginnote} % this is used for notes on the right margin --- \marginnote{\footnotesize txt}

\usepackage{mathtools} % for mathclap command

%\usepackage[normalem]{ulem} % for crossing text out - \sout

% Redefining \def is impossible. I tried, but it is impossible.
%\let\def_prev\def

%%%%%%%%%%%%%%%%%%%%%%%%%%%%%%%%%%%%%%%%%%%%%%%
%           MATH OPERATORS SPACING            %
%%%%%%%%%%%%%%%%%%%%%%%%%%%%%%%%%%%%%%%%%%%%%%%

\let\existstemp\exists
\let\foralltemp\forall
\renewcommand{\exists}{\: \existstemp \:}
\newcommand{\existsonly}{\: \existstemp ! \:}
\renewcommand{\forall}{\: \foralltemp \:}

%%%%%%%%%%%%%%%%%%%%%%%%%%%%%%%%%%%%%%%%%%%%%%%
%            COMMAND SHORTHANDS               %
%%%%%%%%%%%%%%%%%%%%%%%%%%%%%%%%%%%%%%%%%%%%%%%

\newcommand{\example}{{\itshape Пример. }}
\newcommand{\equals}{\Leftrightarrow}
\newcommand{\exc}{{\bfseries Упражнение. }}
\newcommand{\norm}[1]{\left\| #1 \right\|}
\newcommand{\scal}[2]{\left\langle #1, #2 \right\rangle}
\newcommand{\angular}[1]{\langle #1 \rangle}

\newcommand{\Sum}[2]{\underset{#1}{\overset{#2}{\sum}}}
\newcommand{\Int}[2]{\underset{#1}{\overset{#2}{\int}}}
\newcommand{\Ker}{\text{Ker}}

% Physicists' variant of dot product
\newcommand{\pscal}[2]{\, \langle #1 | #2 \rangle \,}
\newcommand{\bra}[1]{\, \langle #1 |}
\newcommand{\ket}[1]{| #1 \rangle \,}

\renewcommand{\leq}{\leqslant}
\renewcommand{\geq}{\geqslant}

%%%%%%%%%%%%%%%%%%%%%%%%%%%%%%%%%%%%%%%%%%%%%%%
%         THEOREM DEFINITION LINES            %
%%%%%%%%%%%%%%%%%%%%%%%%%%%%%%%%%%%%%%%%%%%%%%%

\newtheorem{lem}{Лемма}[section]
\newtheorem{note}{Замечание}[section]
\newtheorem{defi}{Определение}[section]
\newtheorem{theorem}{Теорема}[section]
\newtheorem{state}{Утверждение}[section] % statement

%%%%%%%%%%%%%%%%%%%%%%%%%%%%%%%%%%%%%%%%%%%%%%%
%             GRAPHICS INCLUSION              %
%%%%%%%%%%%%%%%%%%%%%%%%%%%%%%%%%%%%%%%%%%%%%%%

\usepackage{graphicx}

\graphicspath{{./Graphics/}}

%%%%%%%%%%%%%%%%%%%%%%%%%%%%%%%%%%%%%%%%%%%%%%%
%               DRAFT TEMPLATES               %
%%%%%%%%%%%%%%%%%%%%%%%%%%%%%%%%%%%%%%%%%%%%%%%

%\usepackage{marginnotes}
\newcommand{\todo}[1]{\marginpar{\color{red} \tiny #1}}

\usepackage{amscd}

\begin{document}
	\section{Лекция}
	
	В ходе прошлой лекции была доказана теорема Рисса, дающая представление о том, как выглядит функционал в гильбертовом пространстве.
	
	{\color{gray}
		Здесь следует повторная формулировка теоремы Рисса.
	}
	
	Пусть существует функционал $f \in H_2'$. Тогда, по теореме Рисса, ему соответствует вектор $h_f \in H_2$, такой, что 
	$$\forall h \in H_2, \: f(h) = \scal{h}{h_f}$$
	Определим линейный оператор $A : H_1 \rightarrow H_2$. Для $x \in H_1$: 
	$$f(Ax) = \scal{Ax}{h_f}$$
	По теореме Рисса, найдётся единственный вектор $h_g$, такой, что 
	$$f(Ax) = g(x) = \scal{x}{h_g}$$
	И векторы $h_f$ и $h_g$ будут связаны соотношением:
	$$h_g = A^{*}h_f$$
	Таким образом, вводится понятие \textbf{сопряжённого оператора}.

	% Коряво, стоит потом глянуть в литературе.
	Теорема Рисса также приводит нас к доказательству ещё одного важного свойства:
	\begin{state}
		%Пусть функционал $f_1: L \rightarrow \mathbb{C}$ линеен, ограничен и его область определения $D(f_1) = L$,
		%причём $L \neq H$ --- всему гильбертову пространству. Тогда найдётся линейный ограниченный функционал 
		%$f: H \rightarrow \mathbb{C}$, определённый на всём пространстве и $f|_{L} = f_1$.
		Пусть функционал $f$ линеен и ограничен. Если область определения этого функционала $D(f) \subset H$, 
		где $H$ --- гильбертово пространство, то функционал $f$ можно расширить на всё пространство $H$.
	\end{state}
	\todo{Найти эту теорему в прошлых конспектах и отослаться на неё.}
	\begin{proof}
		Ранее было доказано, что линейный ограниченный оператор можно продолжить на замыкание его области определения 
		(Теорема ??). Тогда, даже если $\overline{D(f)} \neq H$, всё равно $\overline{D(f)}$ --- гильбертово пространство.
		А это значит, что, по теореме Рисса, мы можем сопоставить ему вектор $h_f \in \overline{D(f)} \subset H$, такой, что 
		$f(h) = \scal{h}{h_f}$. Но, так как $h_f \in H$, то $\scal{h}{h_f}$ определено на всём пространстве $H$. \\
	\end{proof}
	
	Благодаря этому утверждению, в дальнейшем функционалы можно считать определёнными на всём гильбертовом пространстве.
	
	\subsubsection{Свойства линейного оператора}
	
	\begin{enumerate}
		\item Линейность
		\begin{align*}
			\underline{ \scal{h_1}{A^{*}\angular{\alpha h + \beta g}} } &= \scal{Ah_1}{\alpha h + \beta g} = 
			\bar{\alpha}\scal{Ah_1}{h} \bar{\beta}\scal{Ah_1}{g} = \\
			&= \scal{h_1}{\alpha A^{*} h} + \scal{h_1}{\beta A^{*} g} = 
			\underline{ \scal{h_1}{\alpha A^{*} h + \beta A^{*} g} }
		\end{align*}
		Так как подчёркнутые выражения равны для любых $h_1$, будет выполнено равенство
		$$A^{*}\angular{\alpha h + \beta g} = \alpha A^{*} h + \beta A^{*} g$$
		
		\item Ограниченность \label{bounded}
		
		Рассмотрим скалярное произведение $\scal{h}{A^{*}g}$: 
		$$\scal{h}{A^{*}g} = \scal{Ah}{g} \leq \norm{Ah}\cdot\norm{g} \leq \norm{A}\cdot\norm{h}\cdot\norm{g}$$
		Теперь, поставим $h = A^{*}g$:
		\begin{gather*}
			\norm{A^{*}g}^2 \leq \norm{A}\cdot\norm{A^{*}g}\cdot\norm{g} \\
			\norm{A^{*}g} \leq \norm{A}\cdot\norm{g} \\
			\dfrac{\norm{A^{*}g}}{\norm{g}} \leq \norm{A}
		\end{gather*}
		Вспомним определение нормы оператора: $\norm{A^{*}} = \underset{g \neq 0}{\sup} \frac{\norm{A^{*}g}}{\norm{g}}$. Тогда очевидно
		$\norm{A^{*}} \leq \norm{A}$.
		
		\item $(\alpha A + \beta B)^{*} = (\bar{\alpha}A^{*} + \bar{\beta}B^{*})$ \label{conjlin}
		\begin{align*}
			\scal{(\alpha A + \beta B)h}{g} &= \scal{h}{(\alpha A + \beta B)^{*}g} \\
			\scal{(\alpha A + \beta B)h}{g} &= \alpha\scal{Ah}{g} + \beta\scal{Bh}{g} 
			= \alpha\scal{h}{A^{*}g} + \beta \scal{h}{B^{*}g} = \\
			&= \scal{h}{(\bar{\alpha}A^{*} + \bar{\beta}B^{*})g}
		\end{align*}
		Таким образом, $\forall h\: \scal{h}{(\alpha A + \beta B)^{*}g} = \scal{h}{(\bar{\alpha}A^{*} + \bar{\beta}B^{*})g}$, 
		следовательно $(\alpha A + \beta B)^{*} = (\bar{\alpha}A^{*} + \bar{\beta}B^{*})$ \\
		
		\item $(BA)^{*} = A^{*}B^{*}$
		\todo{Использовать \textbf{tikz-cd}, в нём лучше такие вещи делать.}
		$$
		\begin{CD}
			\scal{BAh}{g} @. = @. \scal{h}{(BA^{*})g} \\
				@| \\
			\scal{Ah}{B^{*}g} @. = @. \scal{h}{A^{*}B^{*}g}
		\end{CD}
		$$
		
		Отсюда $\scal{h}{A^{*}B^{*}g} = \scal{h}{(BA^{*})g}$ и $(BA^{*}) = A^{*}B^{*}$.
		
		{\Large(}Это свойство доказывает $(A^n)^{*} = (A^{*})^n,\, n \in \mathbb{N}$. ($B=A^{n-1}$){\Large)} \\
		
		\item $I^{*} = I$
		$$
		\left.
		\begin{CD}
			\scal{Ih}{g} @. = @. \scal{h}{I^{*}g} \\
				@| \\
			\scal{h}{g} @. = @. \scal{h}{Ig}
		\end{CD}
		\right\} \Rightarrow I = I^{*}
		$$
		
		\item $(A^{*})^{*} = A$ \label{doubleconj}
		
		Для оператора $A: H_1 \rightarrow H_2$, рассмотрим $h \in H_1$, $g \in H_2$:
		$$
			\scal{g}{(A^{*})^{*}h} = \scal{A^{*}g}{h} = \overline{\scal{h}{A^{*}g}} = 
			\overline{\scal{Ah}{g}} = \scal{g}{Ah}
		$$
		Таким образом, $\forall g,h \: \scal{g}{(A^{*})^{*}h} = \scal{g}{Ah} \Rightarrow (A^{*})^{*} = A$
		
		\item $\norm{A} = \norm{A^{*}}$
		
		Очевидно по свойствам (\ref{bounded}) и (\ref{doubleconj}):
		
		$$
		\left.
		\begin{aligned}
			\norm{A^{*}} &\leq \norm{A} \\
			\norm{A} &\leq \norm{A^{*}}
		\end{aligned}
		\right\}
		\Rightarrow \norm{A} = \norm{A^{*}}
		$$
		
		\item Если матрица $A$ обратима, то $(A^{-1})^{*} = (A^{*})^{-1}$.
		
		\exc Доказать это свойство. \\
		{\Large(}Это также автоматически докажет $(A^n)^{*} = (A^{*})^n,\, n \in \mathbb{Z}${\Large)}
	\end{enumerate}
	
	\subsubsection{Самосопряжённый оператор}
	
	\begin{defi}
		Оператор $A : H \rightarrow H$ называется \textbf{самосопряжённым}, если $forall h,\, g \in H$ выполнено выражение
		$$scal{Ah}{g} = \scal{h}{Ag}$$
	\end{defi}
	
	\example Если $\exists H_1 \subset H$, то любой вектор $h$ представим как $h = g+f$, $g \in H_1$, $f\perp H_1$.
	Рассмотрим оператор проецирования на $H_1$ --- $P_{H_1}h = g$. Тогда выполнены следующие равенства:
	$$\scal{P_{H_1}h_1}{h_2} = \scal{h_1}{h_2} = \scal{g_1}{g_2} = \scal{g_1}{P_{H_1}h_2} = \scal{h_1}{P_{H_1}h_2}$$
	Из этой последоватльности равенств следует самосопряжённость оператора $P_{H_1}$.
	
	Для самосопряжённых операторов $A : H \rightarrow H$, $B : H \rightarrow H$ и чисел $\alpha,\, \beta \in \mathbb{R}$ верны следующие
	свойства:
	\begin{enumerate}
		\item $\alpha A + \beta B$ --- самосопряжённый оператор.

		Очевидно из свойства (\ref{conjlin}) сопряжённого оператора.
		
		\item $AB = BA$ $\Rightarrow$ $AB$ --- самосопряжённый оператор.
		
		\todo{А почему тогда только стрелка вправо?}

		На самом деле, два этих утверждения равносильны.
		
		\item $\dfrac{AB - BA}{i}$ --- самосопряжённый.
		
		{\color{gray}
		Доказано на семинаре.
		}
	\end{enumerate}
	Также, для любого ограниченного оператора $A: H \rightarrow H$ верно:
	\begin{enumerate}
		\item[4.] $A+A^{*}$ --- самосопряжённый
		\item[5.] $i(A-A^{*})$ --- самосопряжённый
	\end{enumerate}
	
	\subsubsection{Собственные значения самосопряжённого оператора}
	
	\begin{state}
		Собственные значения для самосопряжённого оператора обязательно вещественны.
	\end{state}
	\begin{proof}
		Рассмотрим собственное значение $\lambda$ и соответствующий ему собственный вектор $e_{\lambda}$:
		\begin{gather*}
			Ae_{\lambda} = \lambda e_{\lambda} \\
			\begin{CD}
				\scal{Ae_{\lambda}}{e_{\lambda}} @.=@. \scal{\lambda e_{\lambda}}{e_{\lambda}} = \lambda \norm{e_{\lambda}}^2 \\
				@| \\
				\scal{e_{\lambda}}{Ae_{\lambda}} @.=@. \scal{e_{\lambda}}{\lambda e_{\lambda}} = \bar{\lambda} \norm{e_{\lambda}}^2 
			\end{CD}
		\end{gather*}
		Так как $\norm{e_{\lambda}} \neq 0$, то $\lambda = \bar{\lambda}$.
	\end{proof}
	\exc Доказать, что для $A = A^{*}$ весь его спектр $\sigma(A) \subset \mathbb{R}$.
	
	\begin{note}
		Собственные вектора самосопряжённого оператора, соответствующие разным собственным значениям, ортогональны.
	\end{note}
	\begin{proof} % или не было.
		Возьмём $\lambda \neq \mu$, причём $\lambda,\, \mu \in \sigma_d(A)$.
		$$
			\begin{CD}
				\scal{Ae_{\lambda}}{e_{\mu}} @.=@. \lambda\scal{e_{\lambda}}{e_{\mu}} \\
				@| \\
				\scal{e_{\lambda}}{Ae_{\mu}} @.=@. \scal{e_{\lambda}}{\mu e_{\mu}} @.=@. \mu\scal{e_{\lambda}}{e_{\mu}}
			\end{CD}
		$$
		Так как $\lambda \neq \mu$, то $\scal{e_{\lambda}}{e_{\mu}} = 0$.
	\end{proof}
	
	\subsubsection{Квадратичная форма оператора}
	\begin{defi}
		Выражение вида $\scal{Ah}{h}$ называется \textbf{квадратичной формой оператора}.
	\end{defi}
	
	\begin{state}
		$$\scal{Ah}{h} = \scal{h}{Ah} = \overline{\scal{Ah}{h}}$$
		Отсюда естественным образом следует $\scal{Ah}{h} \in \mathbb{R}$.
	\end{state}
	
	Стоит также отметить, что $|\scal{Ah}{h}| \leq \norm{A} \cdot \norm{h}^2$ по неравенству Шварца.
	
	Далее в доказательствах будут использоваться два числа:
	$$
		%\left\{
		\begin{aligned}
			m &= \underset{\norm{x} = 1}{\inf} \scal{Ax}{x} \\
			M &= \underset{\norm{x} = 1}{\sup} \scal{Ax}{x}
		\end{aligned}
		%\right.
	$$
	
	\begin{state}
		Если $A$ --- самосопряжённый оператор, то $\norm{A} = \max(|m|, |M|)$. (Норма оператора $A$ равняется супремуму)
	\end{state}
	
\end{document}