\documentclass[12pt]{article}

\usepackage[utf8]{inputenc}
\usepackage[russian]{babel}

\usepackage{amssymb}
\usepackage{amsmath}
\usepackage{amscd}
\usepackage{amsthm}
\usepackage{xcolor}

\usepackage{indentfirst}

%\usepackage{marginnote} % this is used for notes on the right margin --- \marginnote{\footnotesize txt}

\usepackage{mathtools} % for mathclap command

%\usepackage[normalem]{ulem} % for crossing text out - \sout

% Redefining \def is impossible. I tried, but it is impossible.
%\let\def_prev\def

%%%%%%%%%%%%%%%%%%%%%%%%%%%%%%%%%%%%%%%%%%%%%%%
%           MATH OPERATORS SPACING            %
%%%%%%%%%%%%%%%%%%%%%%%%%%%%%%%%%%%%%%%%%%%%%%%

\let\existstemp\exists
\let\foralltemp\forall
\renewcommand{\exists}{\: \existstemp \:}
\newcommand{\existsonly}{\: \existstemp ! \:}
\renewcommand{\forall}{\: \foralltemp \:}

%%%%%%%%%%%%%%%%%%%%%%%%%%%%%%%%%%%%%%%%%%%%%%%
%            COMMAND SHORTHANDS               %
%%%%%%%%%%%%%%%%%%%%%%%%%%%%%%%%%%%%%%%%%%%%%%%

\newcommand{\example}{{\itshape Пример. }}
\newcommand{\equals}{\Leftrightarrow}
\newcommand{\exc}{{\bfseries Упражнение. }}
\newcommand{\norm}[1]{\left\| #1 \right\|}
\newcommand{\scal}[2]{\left\langle #1, #2 \right\rangle}
\newcommand{\angular}[1]{\langle #1 \rangle}

\newcommand{\Sum}[2]{\underset{#1}{\overset{#2}{\sum}}}
\newcommand{\Int}[2]{\underset{#1}{\overset{#2}{\int}}}
\newcommand{\Ker}{\text{Ker}}

% Physicists' variant of dot product
\newcommand{\pscal}[2]{\, \langle #1 | #2 \rangle \,}
\newcommand{\bra}[1]{\, \langle #1 |}
\newcommand{\ket}[1]{| #1 \rangle \,}

\renewcommand{\leq}{\leqslant}
\renewcommand{\geq}{\geqslant}

%%%%%%%%%%%%%%%%%%%%%%%%%%%%%%%%%%%%%%%%%%%%%%%
%         THEOREM DEFINITION LINES            %
%%%%%%%%%%%%%%%%%%%%%%%%%%%%%%%%%%%%%%%%%%%%%%%

\newtheorem{lem}{Лемма}[section]
\newtheorem{note}{Замечание}[section]
\newtheorem{defi}{Определение}[section]
\newtheorem{theorem}{Теорема}[section]
\newtheorem{state}{Утверждение}[section] % statement

%%%%%%%%%%%%%%%%%%%%%%%%%%%%%%%%%%%%%%%%%%%%%%%
%             GRAPHICS INCLUSION              %
%%%%%%%%%%%%%%%%%%%%%%%%%%%%%%%%%%%%%%%%%%%%%%%

\usepackage{graphicx}

\graphicspath{{./Graphics/}}

%%%%%%%%%%%%%%%%%%%%%%%%%%%%%%%%%%%%%%%%%%%%%%%
%               DRAFT TEMPLATES               %
%%%%%%%%%%%%%%%%%%%%%%%%%%%%%%%%%%%%%%%%%%%%%%%

%\usepackage{marginnotes}
\newcommand{\todo}[1]{\marginpar{\color{red} \tiny #1}}

\begin{document}
	Продолжим рассмотрение задачи с незакреплённым правым концом. Используя предыдущие выкладки, получается 
	выражение, из которого можно получить необходимое условие на локальный экстремум:
	$$\Delta I = \int_{x_0}^{x_1} (F_y - \frac{d}{dx} F_{y'})h(x)\,dx + h(x_1)F_{y'}\lims{x=x_1}{} + 
	  F\lims{x=x_1}{}\cdot \delta x_1 + o(\rho(y,\tilde{y}))$$
	  
	Преобразуем $h(x_1)$. Как было записано ранее, $h(x_1) = \bar{y}(x_1) - y(x_1)$
	
	$$\bar{y}(\bar{x}_1) - \bar{y}(x_1) = y'(x_1 + \theta \delta x_1)\cdot \delta x_1 
	= y'(x_1) \delta x_1 + o(\rho(y, \bar{y}))$$
	
	Тогда сможем выразить $\bar{y}$ через $y$:
	$$h(x_1) = \bar{y}(x_1) - y(x_1) = \underbracket{\bar{y}(\bar{x}_1) - y(x_1)}_{\delta y_1}
	  - y'(x_1)\delta x_1 + o(\rho(y, \bar{y}))$$
	\begin{align*}
		\Delta I = \int_{x_0}^{x_1} \left(F_y - \frac{d}{dx}F_{y'}\right)h(x)\,dx
		+ \delta x_1\cdot(F - y'F_{y'}) \lims{x=x_1}{}
		+ \delta y_1 \cdot F_{y'}\lims{x=x_1}{}
	\end{align*}
	
	В науке вариационного исчисления величина $\Delta I$ зачастую называется вариацией функционала. Необходимым условием 
	экстремума будет $\Delta I = 0$. Таким образом, приходим к следующим уравнениям:
	\begin{equation}
	\label{eq:preTraversal}
	\begin{cases}
		&F_y - \frac{d}{dx} F_{y'} = 0 \\
		&(F - y'F_{y'})\lims{x = x_1}{} \delta x_1 + F_{y'} \lims{x=x_1}{}\delta y_1 = 0
	\end{cases} 
	\end{equation}
	
	\off{(Без доказательства)} Если решать задачу где и левый конец функции тоже не закреплён, то $\Delta I$ запишется как:
	\begin{align*}
		\Delta I = \int_{x_0}^{x_1} (F_y - \frac{d}{dx}F_{y'})h(x)\,dx~+~ 
		\delta x\cdot(F - y'F_{y'})\lims{x = x_0}{x = x_1}~+~ \\
		~+~\delta y_1 \cdot F_{y'} \lims{x=x_1}{}-~\delta y_0 \cdot F_{y'} \lims{x=x_0}{}
	\end{align*}
	
	\opt{\off{Популярными граничными условиями являются $F_{y'}\lims{x=x_0}{}=F_{y'}\lims{x=x_1}{}=0$}}

	\todo{Картинка}	
	Если задать, что правый конец экстремали лежит на некоторой функции $\psi(x)$:
	$$\delta y_1 = \psi'(x_1) \delta x_1 + o(\delta x_1)$$
	
	Тогда условие \ref{eq:preTraversal} перепишется как:
	$$
	\begin{cases}
		&F_y - \frac{d}{dx} F_{y'} = 0 \\
		&\delta x_1 \cdot (F - y'F_{y'} + \psi' F_{y'})\lims{x=x_1}{} + o(\delta x_1) = 0
	\end{cases} 
	$$
	
	Уравнение
	$$F - y' F_{y'} + \psi' F_{y'} = 0$$
	называется \textbf{условием траверсальности}.
	
	\exc Рассмотрим функцию вида $F(x, y, y') = f(x,y) \sqrt{1+y'^2}$. Если $F$ имеет такой вид, то $f(x,y)$ имеет простой 
	геометрический смысл и вариационная задача легко разрешима. Проверьте это и покажите, что означает $f(x,y)$ в таком
	случае.
	
	\subsection{Изопериметрическая задача}
	Пусть даны функции 
	$$x(t),\qquad y(t)$$
	с заданными граничными условиями:
	$$x(t_0) = x(t_1) = x_0 \qquad y(t_0) = y(t_1) = y_0$$
	Задача --- максимизировать площадь $S[x,y] = \int_{t_0}^{t_1}x(t)y'(t) dt$ при условии, что периметр
	$l[x,y] = \int_{t_0}^{t_1} \sqrt{\dot{x}^2(t) + \dot{y}^2(t)}\,dt$ фиксирован.
	
	В более общей формулировке задача звучит так: \\
	Найти минимум функционала $I[y] = \int_{x_0}^{x_1} F(x,y,y')\,dx$ если несколько функционалов того же вида
	$J_k[y] = \int_{x_0}^{x_1} G_k(x,y,y')\,dx = C_k$. (Если считать $G_k$ дельта-функцией, то граничные условия
	тоже можно записать в таком виде.)
	
	Для нахождения экстремума в многомерном случае в курсе математического анализа рассматривается равенство
	нулю скалярного произведения градиента на касательную к многообразию.
	Нужно, чтобы производная вдоль любого вектора, касательного к поверхности, 
	\begin{equation}g_i(x) = C_i, \qquad g_1\ldots g_m \label{eq:Surface}\end{equation}
	равнялась нулю. В выбранной точке $x$ касательное пространство является ортогональным дополнением
	к $m$ градиентам из \ref{eq:Surface}. Следовательно, обязательным образом
	$$\nabla f = \sum_i \lambda_i \nabla g_i \quad \Rightarrow \quad f = \sum_i \lambda_i g_i$$
	
	С этими утверждениями из математического анализа вы уже знакомы. Теперь рассмотрим, как они применяются в 
	нашей задаче. Нужно, чтобы вариация функционалов $J_k$ равнялась нулю: $\delta J_k = 0$.
	
	На каждое $G_k$ получается уравнение
	$$\int_{x_0}^{x_1} \left(\underline{G_{k_y} - \frac{d}{dx}G_{k_{y'}}}\right) h(x)\,dx = 0$$	
	где подчёркнутое выражение играет роль градиента. Таким образом,
	$$\delta I = \int_{x_0}^{x_1} \left(F - \frac{d}{dx}F_{y'}\right)h(x) \,dx 
	  = \int_{x_0}^{x_1} \left(\sum_i \left(\lambda_i G_{i_y} - \frac{d}{dx}G_{i_{y'}} \right)\right)h(x) \,dx$$
	  
	Таким образом, должно выполняться условие
	\todo{Наверное, здесь должно быть $h(x)$}
	$$\int_{x_0}^{x_1} \left(F - \sum_i \lambda_i G_i\right) \,dx = 0$$
	
	Ничего не мешает нам считать, что $F(\ldots) = G_0$, тогда $\lambda_0 = 1$. Так как 
	$F$ и $G_i$ --- функции одинакового вида, можно внести $F$ в граничные условия и решать
	задачу для какого-нибудь $G_i$. Эти высказывания формируют \textbf{принцип взаимности}.
	
	\off{Таким образом, изопериметрическая задача равносильна нахождению минимального периметра при 
	фиксированной площади.}
	
	\subsection{Нахождение геодезических кривых}
	
	\begin{defi}
		\textbf{Геодезическая кривая} --- кратчайшая кривая, соединяющая две точки на поверхности.
	\end{defi}
	
	Не будем приводить здесь доказательства рассматриваемого метода, просто представим его <<рецепт>>.
	
	В качестве примера рассмотрим задачу о нахождении геодезических кривых на поверхности сферы.
	Рассматривается функционал следующего вида:
	$$I[x] = \int_{t_0}^{t_1} F(t,x,\dot{x})\,dt \qquad x \in \mathbb{R}^n$$
	То есть существует $n$ функций, описывающих путь по искомой кривой. В исходной задаче такой
	функционал равен $F(t,x,\dot{x}) = \sqrt{\dot{x}^2 + \dot{y}^2 + \dot{z}^2}$, то есть длине
	пути.
	
	Потребуется ввести граничные условия, ограничивающие область поиска:
	\begin{equation} \label{eq:GeodSurface}
		g_i(t, x, \dot{x}) = 0 \qquad i = 1,\ldots k \qquad
		\begin{aligned}
			x(t_0) = x_0 \\
			x(t_1) = x_1
		\end{aligned}
	\end{equation}

	Рассмотрим ещё один функционал:
	
	\todo{Здесь, видимо, тоже должно быть $h(x)$}
	$$J[x] = \int_{t_0}^{t_1} \left(F(t,x,\dot{x}) + \sum_{i=1}^k \lambda_i(t) g_i(t)\right)\,dt$$
	
	Сделаем важное предположение: для $F(\ldots)$, экстремали функционала $I[x]$, обязятельно найдётся
	$k$ функций $\lambda_i(t)$, таких, что решение будет представляться экстремалями функционала $J[x]$.
	При этом, $\lambda_i(t)$ находятся из уравнения Эйлера и граничных условий \ref{eq:GeodSurface}.
	
	\todo{Добавить из семинаров поиск уравнения колебания струны!}
\end{document}