\documentclass[12pt]{article}

\usepackage[utf8]{inputenc}
\usepackage[russian]{babel}

\usepackage{amssymb}
\usepackage{amsmath}
\usepackage{amsthm}
\usepackage{xcolor}

\usepackage{indentfirst}

%\usepackage[normalem]{ulem} % for crossing text out - \sout

% Redefining \def is impossible. I tried, but it is impossible.
%\let\def_prev\def

\newcommand{\example}{{\itshape Пример. }}
\newcommand{\equals}{\Leftrightarrow}
\newcommand{\defi}{{\itshape Определение. }}
\newcommand{\exc}{{\bfseries Упражнение. }}
\newcommand{\norm}[1]{\| #1 \|}

\renewcommand{\leq}{\leqslant}
\renewcommand{\geq}{\geqslant}

%%%%%%%%%%%%%%%%%%%%%%%%%%%%%%%%%%%%%%%%%%%%%%%
%         THEOREM DEFINITION LINES            %
%%%%%%%%%%%%%%%%%%%%%%%%%%%%%%%%%%%%%%%%%%%%%%%

\newtheorem{lem}{Лемма}
\newtheorem{def}{Определение}


\begin{document}
	\title{Основы функционального анализа, второй семестр.}
	\author{Тресков Сергей Андреевич}
	\maketitle
	
	\section{Векторные пространства}
	
    Пора доказать <<контрабанду>>, введенную в прошлом разделе. Докажем, что норма, введенная на основе скалярного произведения, 
    $$\| x \| = \sqrt{<x,x>}$$
    соответствует всем ранее указанным аксиомам.

    \begin{enumerate}
    \item Очевидно из первого свойства скалярного произведения.
    \item $\norm{\lambda x}^2 = \lambda \bar{\lambda} <x, x> = | \lambda |^2 \cdot \norm{x}^2$
    \item $\norm{x+y} \overset{?}{\leq} \norm{x} + \norm{y}$
	      $$\norm{x + y} = \norm{x}^2 + <x,y> + <y,x> + \norm{y}^2 \leq \norm{x}^2 + 2 \norm{x} \norm{y} + \norm{y}^2
			= (\norm{x} + \norm{y})^2
	      $$
    \end{enumerate}
	Таким образом, пространство со скалярным произведением является нормированным пространством и определяет норму на нем.
	
	\begin{def}
		Подмножество векторного пространства называется \textbf{выпуклым}, если оно содержит вместе с любыми двум точками соединяющий
		их отрезок.
	\end{def}
	
\end{document}