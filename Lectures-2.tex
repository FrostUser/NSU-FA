\documentclass[12pt]{article}

\usepackage[utf8]{inputenc}
\usepackage[russian]{babel}

\usepackage{amssymb}
\usepackage{amsmath}
\usepackage{amsthm}
\usepackage{xcolor}

\usepackage{indentfirst}

%\usepackage[normalem]{ulem} % for crossing text out - \sout

% Redefining \def is impossible. I tried, but it is impossible.
%\let\def_prev\def

\newcommand{\example}{{\itshape Пример. }}
\newcommand{\equals}{\Leftrightarrow}
%\newcommand{\defi}{{\itshape Определение. }}
\newcommand{\exc}{{\bfseries Упражнение. }}
\newcommand{\norm}[1]{\| #1 \|}

\renewcommand{\leq}{\leqslant}
\renewcommand{\geq}{\geqslant}

%%%%%%%%%%%%%%%%%%%%%%%%%%%%%%%%%%%%%%%%%%%%%%%
%         THEOREM DEFINITION LINES            %
%%%%%%%%%%%%%%%%%%%%%%%%%%%%%%%%%%%%%%%%%%%%%%%

\newtheorem{lem}{Лемма}[section]
\newtheorem{defi}{Определение}[section]
\newtheorem{theorem}{Теорема}[section] % statement
\newtheorem{state}{Утверждение}[section] % statement
%\newtheorem{proof}{Определение}


\begin{document}
	\title{Основы функционального анализа, второй семестр.}
	\author{Тресков Сергей Андреевич}
	\maketitle
	
	\section{Векторные пространства}
	
    Пора доказать <<контрабанду>>, введенную в прошлом разделе. Докажем, что норма, введенная на основе скалярного произведения, 
    $$\| x \| = \sqrt{<x,x>}$$
    соответствует всем ранее указанным аксиомам.

    \begin{enumerate}
    \item Очевидно из первого свойства скалярного произведения.
    \item $\norm{\lambda x}^2 = \lambda \bar{\lambda} <x, x> = | \lambda |^2 \cdot \norm{x}^2$
    \item $\norm{x+y} \overset{?}{\leq} \norm{x} + \norm{y}$
	      $$\norm{x + y} = \norm{x}^2 + <x,y> + <y,x> + \norm{y}^2 \leq \norm{x}^2 + 2 \norm{x} \norm{y} + \norm{y}^2
			= (\norm{x} + \norm{y})^2
	      $$
    \end{enumerate}
	Таким образом, пространство со скалярным произведением является нормированным пространством и определяет норму на нем.
	
	\begin{defi}
		Подмножество векторного пространства называется \textbf{выпуклым}, если оно содержит вместе с любыми двумя 
		точками соединяющий их отрезок.
	\end{defi}

	В доказательстве последующей теоремы потребуется данное утверждение:
	\begin{state}
		Скалярное произведение непрерывно по первому и второму аргументам.
		$$<x, y> - <x', y'> = <x, y> - <x', y> + <x', y> - <x', y'> \leq$$
		$$\leq \norm{x - x'}\cdot \norm{y} + \norm{x'} \cdot \norm{y - y'}$$
	\end{state}
	
	\begin{theorem}
		Пусть G' --- замкнутое выпуклое подмножество H. H - гильбертово пространство. Тогда
		$$\forall h \in H \exists! g \in G \norm{h-g} = \underset{g' \in G}{inf} \norm{h-g'} = \alpha$$
	\end{theorem}
	\begin{proof}
		Рассмотрим последовательность $g_1, g_2, ..., g_n$ $\norm{h - g_n} \rightarrow \alpha$
		$$\forall \epsilon \exists N(\epsilon), n > N(\epsilon) \norm{h - g_n} < \alpha + \epsilon$$
		Примем $n, m > N(\epsilon)$. Рассмотрим вектора $h - g_n$, $h - g_m$, запишем для них тождество параллелограмма.
		$$\norm{g_n - g_m} = 2 \cdot \norm{h - g_n} + 2 \cdot \norm{h - g_m}$$
		$$\norm{g_n - g_m}^2 = 2 \cdot \norm{h - g_n}^2 + 2 \cdot \norm{h - g_m}^2 - \norm{2h - (g_n + g_m)}^2$$
		$$\norm{g_n - g_m}^2 = 2 \cdot \norm{h - g_n}^2 + 2 \cdot \norm{h - g_m}^2 - 4 \cdot \norm{h - \frac{g_n + g_m}{2}}^2 \leq$$
		В силу выпуклости подмножества, $\frac{g_n + g_m}{2} \in G$, что означает $\norm{h - \frac{g_n + g_m}{2}} \geq \alpha$, так
		как $\alpha$ - точная нижняя грань.
		$$\leq 2 (\alpha + \epsilon)^2 + 2 (\alpha + \epsilon)^2 - 4 \alpha^2 = 8 \alpha \epsilon + 4 \epsilon^2$$
		Из этого выражения получаем, что последовательность $\norm{h - g_n}$ --- фундаментальная, следовательно найдется $g_0$, такое,
		что $\norm{h - g_0} = \alpha$.
		
		Также докажем единственность. Пусть таких векторов найдется два: g' и g''. Тогда
		$$\norm{h - g'} - \norm{h - g''} = \alpha$$
		$$\norm{g' - g''}^2 = 2 \norm{h - g'}^2 + 2 \norm{h - g''}^2 - 4 \norm{h - \frac{g' + g''}{2}} \leq$$
		$$\leq 4 \alpha^2 - 4 \alpha^2 = 0$$
		$\Rightarrow g' = g'' \Rightarrow$ такой вектор найдется только один.
	\end{proof}
	
	\begin{theorem}
		G - замкнутое подпространство H. $h \in H, g \in G$, g --- ближайший. Тогда $f = (h - g) \perp G$
	\end{theorem}
	Мы можем выделить ближайший вектор, так как подпространство обязательно является выпуклым. {\color{gray} Деваться больше некуда.}
	\begin{proof}
		Предположим, что это не так:
		$$<f, g_1> = \alpha \neq 0$$
		Рассмотрим вектор:
		$$g^* = g + \frac{<f, g_1>}{\norm{g1}^2} \cdot g_1 = $$
		$$ = g + \frac{\alpha}{\norm{g_1}^2} \cdot g_1$$
		Используя этот вспомогательный вектор, докажем ортогональность:
		$$\norm{h - g^*}^2 = <h - g - \frac{\alpha}{\norm{g_1}^2} \cdot g_1, \underset{То же самое}{...}> = $$
		$$ = \norm{h - g}^2 - <h - g, \frac{\alpha}{\norm{g_1}^2} g_1> - <\frac{\alpha}{\norm{g_1}^2} g_1, h - g> + 
		\frac{|\alpha|^2}{\norm{g_1}^2} g_1 = $$
		$$ = \norm{h - g}^2 - \frac{|\alpha|^2}{\norm{g_1}^2} - \frac{|\alpha|^2}{\norm{g_1}^2} + \frac{|\alpha|^2}{\norm{g_1}^2} = $$
		$$ = \norm{h - g}^2 - \frac{|\alpha|^2}{\norm{g_1}^2}$$
		То есть получается, что $g^*$ ближе к h, чем g, что невозможно из предыдущей теоремы. Отсюда следует $\alpha = 0$.
	\end{proof}
	
	Таким образом, любой вектор гильбертова пространства представляется в виде суммы двух векторов:
	$g \in G$ (G --- замкнутое подпространство H) и $g_2 \in G^\perp$ ($G^\perp$ --- ортогональное дополнение к G).
	А это значит, что $H = H_1 \oplus H_1^\perp$
	
	Пусть существует ортонормированная последовательность $e_1, e_2, ..., e_n, ...$. Рассмотрим первые n элементов.
	$E_n = \{ \sum_1^n \alpha_i e_i \}$
	
	Значит для $h \in H, \exists g_n \in E_n$, такое что $f_n = h - g_n$ и $f \perp E_n$. \\
	$g_n = \sum_1^n \alpha_i e_i, \alpha_i = <h, e_i>$ \\
	Для $h = g + f$, (так как $<g,f> = 0$)
	$$ \norm{h}^2 = \norm{g_n}^2 + \norm{f_n}^2 = \sum_1^n |\alpha_i|^2 + \norm{f_n}^2$$
	Имеет смысл рассмотреть такую величину в гильбертовом пространстве: $h = \sum_1^n \alpha_i e_i, \alpha_i = <h e+i>$ -
	\textbf{ряд Фурье} вектора h.
	$$ \norm{h}^2 \geq \sum_1^n |\alpha_i|^2 = \norm{g_n}^2 $$

	%%	Точно ли в выражении выше стоит \geq? Это стоит проверить.
	
	Таким образом, мы вывели \textbf{неравенство Бесселя}:
	%$$ \norm{h}^2 \geq \sum_1^\inf |\alpha_i|^2 $$

	Это неравенство гарантирует сходимость ряда $\sum_1^{\inf} |\alpha_i|^2$, что обеспечивает фундаментальность 
	последовательности векторов $g_n$.
	
	\begin{defi}
		Ортонормированную систему будем называть полной, если её нальзя пополнить (то есть добавить единичный вектор $e_{n+1}$, 
		перпендикулярный предыдущим).
	\end{defi}
	\begin{defi}
		Ортонормированную систему будем называть замкнутой, если для любого вектора из гильбертова пространства неравенство Бесселя
		становится равенством.
	\end{defi}
	
	\begin{state}
		Пусть система векторов $\{ e_\alpha \}$ полна, тогда $\overline{ \nu \{ e_\alpha \} } = H$.
	\end{state}
	\begin{proof}
		Предположим, что это не так. Пусть $h_0 \notin \overline{ \nu \{ e_\alpha \} }$, тогда найдется ближайший вектор $\Rightarrow 
		\exists f_0 	= (h_0 - g_0)^\perp	\Rightarrow$ систему $\{ e_{\alpha} \}$ можно пополнить.
	\end{proof}
	
	{\huge НЯ!}
\end{document}