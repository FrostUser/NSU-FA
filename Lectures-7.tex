\documentclass[12pt]{article}

\usepackage[utf8]{inputenc}
\usepackage[russian]{babel}

\usepackage{amssymb}
\usepackage{amsmath}
\usepackage{amscd}
\usepackage{amsthm}
\usepackage{xcolor}

\usepackage{indentfirst}

%\usepackage{marginnote} % this is used for notes on the right margin --- \marginnote{\footnotesize txt}

\usepackage{mathtools} % for mathclap command

%\usepackage[normalem]{ulem} % for crossing text out - \sout

% Redefining \def is impossible. I tried, but it is impossible.
%\let\def_prev\def

%%%%%%%%%%%%%%%%%%%%%%%%%%%%%%%%%%%%%%%%%%%%%%%
%           MATH OPERATORS SPACING            %
%%%%%%%%%%%%%%%%%%%%%%%%%%%%%%%%%%%%%%%%%%%%%%%

\let\existstemp\exists
\let\foralltemp\forall
\renewcommand{\exists}{\: \existstemp \:}
\newcommand{\existsonly}{\: \existstemp ! \:}
\renewcommand{\forall}{\: \foralltemp \:}

%%%%%%%%%%%%%%%%%%%%%%%%%%%%%%%%%%%%%%%%%%%%%%%
%            COMMAND SHORTHANDS               %
%%%%%%%%%%%%%%%%%%%%%%%%%%%%%%%%%%%%%%%%%%%%%%%

\newcommand{\example}{{\itshape Пример. }}
\newcommand{\equals}{\Leftrightarrow}
\newcommand{\exc}{{\bfseries Упражнение. }}
\newcommand{\norm}[1]{\left\| #1 \right\|}
\newcommand{\scal}[2]{\left\langle #1, #2 \right\rangle}
\newcommand{\angular}[1]{\langle #1 \rangle}

\newcommand{\Sum}[2]{\underset{#1}{\overset{#2}{\sum}}}
\newcommand{\Int}[2]{\underset{#1}{\overset{#2}{\int}}}
\newcommand{\Ker}{\text{Ker}}

% Physicists' variant of dot product
\newcommand{\pscal}[2]{\, \langle #1 | #2 \rangle \,}
\newcommand{\bra}[1]{\, \langle #1 |}
\newcommand{\ket}[1]{| #1 \rangle \,}

\renewcommand{\leq}{\leqslant}
\renewcommand{\geq}{\geqslant}

%%%%%%%%%%%%%%%%%%%%%%%%%%%%%%%%%%%%%%%%%%%%%%%
%         THEOREM DEFINITION LINES            %
%%%%%%%%%%%%%%%%%%%%%%%%%%%%%%%%%%%%%%%%%%%%%%%

\newtheorem{lem}{Лемма}[section]
\newtheorem{note}{Замечание}[section]
\newtheorem{defi}{Определение}[section]
\newtheorem{theorem}{Теорема}[section]
\newtheorem{state}{Утверждение}[section] % statement

%%%%%%%%%%%%%%%%%%%%%%%%%%%%%%%%%%%%%%%%%%%%%%%
%             GRAPHICS INCLUSION              %
%%%%%%%%%%%%%%%%%%%%%%%%%%%%%%%%%%%%%%%%%%%%%%%

\usepackage{graphicx}

\graphicspath{{./Graphics/}}

%%%%%%%%%%%%%%%%%%%%%%%%%%%%%%%%%%%%%%%%%%%%%%%
%               DRAFT TEMPLATES               %
%%%%%%%%%%%%%%%%%%%%%%%%%%%%%%%%%%%%%%%%%%%%%%%

%\usepackage{marginnotes}
\newcommand{\todo}[1]{\marginpar{\color{red} \tiny #1}}

\setcounter{section}{7}

\begin{document}
	\title{Основы функционального анализа, второй семестр.}
	\author{Тресков Сергей Андреевич}
	\maketitle
	
	% Упоминание того, что было в прошлой лекции.
	
	\subsection*{Спектр и регулярные значения линейного оператора}
	
	На прошлой лекции был рассмотрен оператор $(A - \lambda I)$, и классификация значений $\lambda$, определяющаяся свойствами этого 
	оператора. Для удобства, в дальнейшем будем использовать обозначение $A_{\lambda}$.
	
	\begin{state}
		Множество регулярных значений $r(A)$ оператора $A$ является открытым.
	\end{state}
	\begin{proof}
		Ранее было доказано, что оператор $(A + \Delta)$ обратим, если $\norm{\Delta} \leq \frac{1}{\norm{A^{-1}}}$ 
		и $A$ является обратимым оператором. Тогда, если $\lambda \in r(A)$, то оператор
		$$ (A - \lambda I) - \mu I $$
		является обратимым для малых $\mu$. Этим доказывается открытость множества $r(A)$.
	\end{proof}
	С другой стороны, для любого линейного оператора верно
	$$
		\overbrace{r(A)}^{ \mathclap{\text{Регулярные значения $A$}} } \cup 
		\underbrace{\sigma_1 (A) \cup \sigma_2 (A) \cup \sigma_3 (A)}_{\text{Спектр $A$}} = 
		r(A) \cup \sigma(A) = \mathbb{C}
	$$
	Данное равенство означает, что, в силу открытости $r(A)$, \textbf{спектр линейного оператора --- закрытое множество}.
	
	\subsection*
	{
		Сопряжённые гильбертовы пространства. \\
		Основная\footnote{В рамках нашего курса.} теорема гильбертова пространства.
	}
	
	\begin{defi}
		Пусть $E$ --- банахово пространство. Тогда $E'$ будем называть 
		\textbf{пространством непрерывных линейных функционалов из $E$ в $\mathbb{C}$} 
		или \textbf{пространством, сопряжённым к $E$}.
	\end{defi}
	
	Данное определение очень похоже на определение обобщённых функций, введённых в предыдущем курсе функционального анализа. 
	Единственным отличием является то, что теперь функционалы задаются в нормированном пространстве, благодаря в отличие от
	обобщённых функций, где топология задавалась исключительно понятием сходимости
	\footnote
	{
		Строго говоря, топология пространства обобщённых функций также может определяться множеством 
		полунорм --- норм, которые могут равняться нулю на ненулевых элементах. Но эта информация выходит
		за рамки данного курса.
	}
	.
	
	Как и для линейных операторов, линейный функционал непрерывен тогда и только тогда, когда он ограничен. Единственное отличие
	между ними заключается в том, что линейные функционалы обладают фиксированным множеством прибытия.
	
	Преимущественно будем использовать $H'$ --- пространство, сопряжённое гильбертову.
	
	Рассмотрим линейный функционал $l: H \rightarrow \mathbb{C}$. Для него определено понятие ядра:
	$$\Ker(l) = \{ h \in H | l \angular{h} = 0\}$$
	В общем случае непрерывного линейного функционала, ядро обладает следующими свойствами:
	\begin{itemize}
		\item $\Ker$ --- не пустое множество. \\
		(В силу линейности в нём обязательно лежит $h = 0$)
		\item $\Ker$ --- линейное подпространство. \\
		(В силу линейности функционала)
		\item $\Ker$ --- замкнутое подпространство. \\
		(Пусть $x_n \in \Ker(l)$, $x_n \rightarrow x_0 \Rightarrow l\angular{x_n} \rightarrow l\angular{x_0}$)
	\end{itemize}
	
	\exc Доказать, что если ядро линейного функционала $f$ замкнуто, то $f$ ограничен.
	
	% TODO: Разобрать, как поставить, чтобы была не теорема 4.1, а теорема Рисса.
	\begin{theorem} 
		Пусть $l H \rightarrow \mathbb{C}$ - линейный ограниченнный функционал, тогда
		\begin{enumerate}
			\item $\existsonly h_l \in H$, такой, что 
			$$\forall h \in H, l \angular{h} = \scal{h}{h_l}$$
			\item $\forall g \in H$, формула (markme!) определяет линейный непрерывный функционал
			$$f\angular{h} \rightarrow \mathbb{C}$$
			$$f\angular{h} = \scal{h}{g}$$
		\end{enumerate}
	\end{theorem}
	\begin{proof}
		Докажем по порядку приведённые пункты теоремы:
		\begin{enumerate}
			\item Обозначим $G = \Ker(l)$. В таком случае, возможны два варианта:
			\begin{enumerate}
				\item Ядро совпадает с гильбертовым пространством: $(G = H)$. \\
				Данное равенство будет означать, что любой вектор пространства обращается функционалом в ноль.
				Следовательно, $h_l = 0$ будет единственным подходящим решением.
				\item Ядро не совпадает с гильбертовым пространством: $G \neq H$. \\
				Так как $G$ --- замкнутое подпространство, то его ортогональное дополнение $G^{\perp} \neq \varnothing$
				
				Тогда можно взять вектор $h_0 \in G^{\perp}, h_0 \neq 0$. Взяв произвольный $h \in H$, рассмотрим вектор
				\begin{equation} \label{eq:RissVector}
					(l\angular{h}) h_0 - (l\angular{h_0}) h
				\end{equation}
				Несложно показать, что вектор \eqref{eq:RissVector} лежит в G. Действительно,
				$$
					l\angular{(l\angular{h}) h_0 - (l\angular{h_0}) h} 
					= l\angular{h}l\angular{h_0} - l\angular{h_0}l\angular{h} = 0
				$$
				Теперь рассмотрим скалярное произведение векторов \eqref{eq:RissVector} и $h_0$:
				$$
					(l\angular{h}) \cdot \norm{h_0}^2 - (l\angular{h_0}) \cdot \scal{h}{h_0} = 0
				$$
				Откуда в результате нехитрых преобразований получается 
				$$ h_l = \frac{\overline{l\angular{h_0}}}{\norm{h_0}^2} \cdot h_0 $$
			\end{enumerate}
			Докажем единственность полученного $h_l$. Предположим, что это не так и существуют два вектора $h'$ и $h''$, таких, что 
			$$l\angular{h} = \scal{h}{h'} = \scal{h}{h''}$$
			Тогда будет верно
			\begin{eqnarray*}
				\scal{h}{h'} = \scal{h}{h''} \\
				\scal{h}{(h' - h'')} = 0
			\end{eqnarray*}
			Так как данное равенство верно для любых $h$, возьмём $h = h' - h''$. Получим $\norm{h'-h''}^2 = 0$, откуда следует
			$h' = h''$, что и требовалось доказать.
			
			\item Рассмотрим функцию $f$, такую, что $f(h) = \scal{h}{g}$
		\end{enumerate}
	\end{proof}
\end{document}