\documentclass[12pt]{article}

\usepackage[utf8]{inputenc}
\usepackage[russian]{babel}

\usepackage{amssymb}
\usepackage{amsmath}
\usepackage{amscd}
\usepackage{amsthm}
\usepackage{xcolor}

\usepackage{indentfirst}

%\usepackage{marginnote} % this is used for notes on the right margin --- \marginnote{\footnotesize txt}

\usepackage{mathtools} % for mathclap command

%\usepackage[normalem]{ulem} % for crossing text out - \sout

% Redefining \def is impossible. I tried, but it is impossible.
%\let\def_prev\def

%%%%%%%%%%%%%%%%%%%%%%%%%%%%%%%%%%%%%%%%%%%%%%%
%           MATH OPERATORS SPACING            %
%%%%%%%%%%%%%%%%%%%%%%%%%%%%%%%%%%%%%%%%%%%%%%%

\let\existstemp\exists
\let\foralltemp\forall
\renewcommand{\exists}{\: \existstemp \:}
\newcommand{\existsonly}{\: \existstemp ! \:}
\renewcommand{\forall}{\: \foralltemp \:}

%%%%%%%%%%%%%%%%%%%%%%%%%%%%%%%%%%%%%%%%%%%%%%%
%            COMMAND SHORTHANDS               %
%%%%%%%%%%%%%%%%%%%%%%%%%%%%%%%%%%%%%%%%%%%%%%%

\newcommand{\example}{{\itshape Пример. }}
\newcommand{\equals}{\Leftrightarrow}
\newcommand{\exc}{{\bfseries Упражнение. }}
\newcommand{\norm}[1]{\left\| #1 \right\|}
\newcommand{\scal}[2]{\left\langle #1, #2 \right\rangle}
\newcommand{\angular}[1]{\langle #1 \rangle}

\newcommand{\Sum}[2]{\underset{#1}{\overset{#2}{\sum}}}
\newcommand{\Int}[2]{\underset{#1}{\overset{#2}{\int}}}
\newcommand{\Ker}{\text{Ker}}

% Physicists' variant of dot product
\newcommand{\pscal}[2]{\, \langle #1 | #2 \rangle \,}
\newcommand{\bra}[1]{\, \langle #1 |}
\newcommand{\ket}[1]{| #1 \rangle \,}

\renewcommand{\leq}{\leqslant}
\renewcommand{\geq}{\geqslant}

%%%%%%%%%%%%%%%%%%%%%%%%%%%%%%%%%%%%%%%%%%%%%%%
%         THEOREM DEFINITION LINES            %
%%%%%%%%%%%%%%%%%%%%%%%%%%%%%%%%%%%%%%%%%%%%%%%

\newtheorem{lem}{Лемма}[section]
\newtheorem{note}{Замечание}[section]
\newtheorem{defi}{Определение}[section]
\newtheorem{theorem}{Теорема}[section]
\newtheorem{state}{Утверждение}[section] % statement

%%%%%%%%%%%%%%%%%%%%%%%%%%%%%%%%%%%%%%%%%%%%%%%
%             GRAPHICS INCLUSION              %
%%%%%%%%%%%%%%%%%%%%%%%%%%%%%%%%%%%%%%%%%%%%%%%

\usepackage{graphicx}

\graphicspath{{./Graphics/}}

%%%%%%%%%%%%%%%%%%%%%%%%%%%%%%%%%%%%%%%%%%%%%%%
%               DRAFT TEMPLATES               %
%%%%%%%%%%%%%%%%%%%%%%%%%%%%%%%%%%%%%%%%%%%%%%%

%\usepackage{marginnotes}
\newcommand{\todo}[1]{\marginpar{\color{red} \tiny #1}}

\begin{document}

\section{Интегральные уравнения}

	\subsection{Уравнения Фредгольма и Вольтерра}

		Далее некоторая часть нашего курса будет посвящена решению интегральных уравнений. Введём несколько определений:
	
		\begin{defi}
			\textbf{Уравнениями Фредгольма} называются следующие уравнения:
			\begin{align*}
				&(Ax)(t) + f(t) = 0 &\qquad &\text{Уравнение Фредгольма I рода} \\
				&(Ax)(t) + f(t) = x(t) &\qquad &\text{Уравнение Фредгольма II рода}
			\end{align*}
			Где $A$ --- оператор Гильберта-Шмидта.
		\end{defi}
	
		\todo{Здесь было ещё одно утверждение, которое желательно уточнить.}
		% $t > x,\, K(x,t) = 0$
		Преимущественно мы будем рассматривать ситуацию 
		$$\supp K(x,t) \subset [a,b]^2$$
		Что означает
		$$A_K: \mathbb{L}_2(a,b) \rightarrow \mathbb{L}_2(a,b)$$
	
		Зачастую будет рассматриваться \textbf{уравнение Вольт\'eрра} II рода:
		$$\int_a^t K(t,u) x(u) du + f(t) = x(t)$$
		Это уравнение, как будет доказано на следующей лекции, является частным случаем уравения Фредгольма 
		II рода. Это уравнение рассматривается отдельно, так как для него доказаны существование и единственность
		решения.

		\lecture{11}

		\opt{
			Несколько слов о теореме Фредгольма. В нашем курсе, она доказывается для гильбертовых пространств.
			В общем же случае, при формулировке для банаховых пространств, не получится использовать понятие
			ортогональности --- в банаховом пространстве не обязательно есть скалярное произведение. \par
		
			Но, строго говоря, был не доказан ещё один факт. В рассматриваемой нами теореме Фредгольма 
			для гильбертовых пространств $\Im(I - A)^{\perp} = \Ker(I - A^*)$. Известно, что $\Ker(I-A^*)$ 
			замкнуто. Таким образом, требуется также доказать замкнутость образа $(I-A)$. (Для компактного
			оператора это верно, так что всё в порядке)
		}

	\subsection{Собственные числа интегрального оператора. Повторные ядра}
	
		Рассмотрим следующее уравнение:
	
		\begin{equation*}	
			\mu \underbrace{\int_a^b k(t,s) x(s)\,ds}_{(Ax)(t)} + f(t) = x(t) \\		
		\end{equation*}
		\begin{equation}
			(I - \mu A)x = f \label{eq:smallParam}
		\end{equation}
	
		По теореме, рассматриваемой ранее, для $\mod{\mu} \leq \frac{1}{\norm{A}}$, 
		уравнение (\ref{eq:smallParam}) будет разрешимо.
	
		\begin{defi}
			Пусть $M$ --- полное метрическое пространство. Отображение $g: M\rightarrow M$ называется \textbf{сжимающим}, если
			$$\exists 0 < C < 1 \: \forall x_1,x_2 \in M\!\!:\: \rho(g(x_1), g(x_2)) \leq C\cdot\rho(x_1,x_2)$$
		\end{defi}
	
		\begin{theorem} \label{th:CompFunc}
			Пусть $g$ --- сжимающее отображение. Тогда оно имеет единственную \textbf{неподвижную точку} $x_0$, 
			то есть такую, что $g(x_0) = x_0$.
		\end{theorem}
	
		\begin{proof}
			\textbf{Единственность}: предположим, что таких точек найдётся две: $\tilde{x}$ и $\bar{x}$. Рассмотрим 
			расстояние между образами этих точек.
		
			\begin{align*}
				\rho(g(\tilde{x})&, g(\bar{x})) \leq C \cdot \rho(\tilde{x}, \bar{x}) \\
				&\parallel \\
				\rho(&\tilde{x}, \bar{x})
			\end{align*}
		
			Тогда получается $C \geq 1$, что противоречит определению сжимающего отображения. Следовательно, если такая точка есть, она
			будет единственной.
		
			\textbf{Существование}: Рассмотрим последовательность 
			$$x_1, g(x_1) = x_2, g(x_2) = x_3, \ldots$$
			Тогда $\rho(x_{n+1},x_{n+2}) \leq C \cdot \rho(x_n, x_{n+1})$. Для точек $x_{n+k}$ и $x_{n+k+1}$ получим
		
			\begin{align*}
				\rho(x_{n+k},x_{n+k+1}) &\leq \rho(x_n, x_{n+1}) \cdot C^k \\
				\rho(x_{n+k},x_{n+1}) &\leq \underbracket{\rho(x_n, x_{n+1})}_{\leq \rho(x_1, x_2)\cdot C^{n-1}} 
				\cdot \underbracket{(1 + C + \ldots + C^k)}_{\leq \frac{1	}{1-C}} \\
				\rho(x_{n+k},x_{n+1}) &\leq \dfrac{\rho(x_1, x_2)\cdot C^{n-1}}{1-C}
			\end{align*}
		
			Расстояние между элементами последовательности постоянно уменьшается. Значит, для любого $\varepsilon > 0$, мы 
			всегда можем выбрать $n_0$, при котором $\rho(x_n, x_m) < \varepsilon$, для $n,m > n_0$. Значит, последовательность
			$\{x_n\}$ фундаментальна. Так как пространство, которое мы рассматриваем, полное, она будет иметь предел.		
			$$\underset{n \rightarrow \infty}{\lim} x_n = x_0$$		
			Непрерывность $g(\ldots)$ означает, что 
			$$\underset{n \rightarrow \infty}{\lim} g(x_n) = g(x_0)$$
		
			Таким образом, $g(x_0) = x_0$ --- стационарная точка и она существует для любого сжимающего отображения.
		\end{proof}
	
		Какое же отношение эти сжимающие отображения имеют к уравнению \eqref{eq:smallParam} с малым параметром? 
		Сейчас мы это узнаем. Введём 
		\begin{equation}
			g\big(x(t)\big) = \big((\mu A)x\big)(t) + f(t) \label{eq:compactingFunction}
		\end{equation}
	
		Поймём, что $g(x)$ --- сжимающее отображение.
	
		$$\norm{g(x_1) - g(x_2)} = \norm{\mu A(x_1 - x_2)} \leq \mod{\mu} \norm{A} \cdot \norm{x_1 - x_2}$$
	
		Следовательно, $g(x)$ будем являться сжимающим отображением при $\mu \leq \frac{1}{\norm{A}}$. При этом
		неподвижная точка $g(x_0) = x_0$ будет являться решением уравнения \eqref{eq:smallParam}.
	
		\begin{defi}
			Число $\mu$, при которых оператор $(I - \mu A)$ не обратим, называются \textbf{собственным числом
			интегрального оператора}.
		\end{defi}
	
		В частности, <<по вине>> этого определения теорема Гильберта-Шмидта для интегральных операторов формируется
		чуть-чуть по-другому.
	
		Если определена композиция операторов Гильберта-Шмидта (получится повторный интеграл, который надо будет 
		свести к двойному) тоже является оператором Гильбрета-Шмидта и ядро композиции операторов выражается 
		через ядра этих операторов. Тогда, если $A$ --- оператор Гильберта-Шмидта, то $A^n$ тоже будет являться
		таковым.
	
		\begin{defi}
			Пусть $A$ --- оператор Гильберта-Шмидта. Тогда ядро оператора $A^n$ называется \textbf{повторным ядром}.
		\end{defi}
	
		{\footnotesize
		Формула повторного ядра будет получена на семинаре. Кстати говоря, в книге Александрова <<Интегральные уравнения>>
		приведена запись резольвентного ядра. Если исправить в ней некую ошибку, эта запись может оказаться крайне полезной
		в решении семестровых заданий.\par
		}

	\subsection{Решение уравнения Вольтерра}

		Рассмотрим последовательность для отображения \eqref{eq:compactingFunction}, подобную последовательности
		из теоремы \eqref{th:CompFunc} о стационарной точке для сжимающего отображения.
	
		$$0, \underbrace{f}_{g(0)}, \underbrace{\mu Af + f}_{g(f)}, \ldots, f + \sum_{n=1}^N \mu^nA^nf \rightarrow \sum_0^{\infty} \mu^nA^n f$$

		С помощью данной последовательности в теореме \eqref{th:CompFunc} мы обнаружили неподвижную точку
		сжимающего отображения. {\color{gray} Это и логично: отображение сжимающее, следовательно оно будет
		всё ближе и ближе <<сжиматься>> к неподвижной точке, чем чаще мы его применяем.} Следовательно, можем
		записать:

		$$x = \sum_0^{\infty} \mu^nA^n f$$

		Единственность неподвижной точки доказана, значит решение (если мы докажем, что это $x$ ---
		действительно решение) уравнения Вольтерра единственно.

		Всё, что требуется нам доказать, --- это существование этого решение, а для этого достаточно показать
		сходимость ряда $\sum_0^{\infty} \mu^nA^n f$. На этом и базируется доказательство теоремы о
		существовании и единственности решения уравнения Вольтерра.
	
		\begin{theorem}
			(О существовании и единственности решения уравнения Вольтерра)
			Уравнение Вольтерра
			$$x(t) = \underbrace{\int_a^t k(t,s) x(s) \, ds}_{(Ax)(t)} + f(t) \qquad t\in [a,b]$$
			имеет единственное решение, которое может быть найдено методом последовательных приближений.
		\end{theorem}		
	
		\begin{proof}
			Рассмотрим ряд ($I + A + A^2 + \ldots$). Если взять первые $n$ слагаемых этого ряда и умножить их на $(I - A)$, 
			получим следующее выражение:
			$$(I-A) \cdot \sum_{k=0}^n A^k = I - A + A - A^2 + A^2 - A^3 \pm\ldots = I - A^n \rightarrow I$$
			$$(I-A) \cdot \sum_{k=0}^\infty A^k = I ~ \Rightarrow ~ \sum_{k=0}^\infty A^k = (I-A)^{-1}$$

			Значит, если этот ряд сходится, то он будет сходиться к $(I - A)^{-1}$. Но сходимость не доказана.
		
			Оператор $(Ax)(t)$ можно рассматривать как скалярное произведение в $\mathbb{L}_2$. Тогда, из неравенства Шварца:
			$$\mod{(Ax)(t)}^2 \leq 
			  \int_a^t \mod{k(t,s)}^2 \,ds \cdot \int_a^t \mod{x(s)}^2 \,ds \leq \ldots$$
			  
			Сделаем предположение, сильно упрощающее доказательство теоремы. Пусть ядро оператора $A$ 
			ограничено: $\mod{k(t,s)} \leq M$. Продолжим цепочку неравенств:
		
			$$\ldots \leq M^2(t-a) \int_a^b \mod{x(s)}^2 \,ds = M^2(t-a)\norm{x}^2$$
		
			Таким образом, оператор $A$ ограничен. Рассмотрим теперь $A^2$.
		
			\begin{align*}
				{(A^2x)(t)}^2 = \mod{\int_a^t k(t,s)(Ax)(s) \,ds} \leq M^2(t-a)\int_a^t\mod{(Ax)(s)}^2\,ds \leq \\ 
				\leq M^2(t-a)\int_a^tM^2(s-a)\norm{x}^2 \,ds = M^4 \norm{x}^2 \frac{(t-a)^2}{2}(t-a)
			\end{align*}
		
			Таким образом, $A^2$ --- тоже ограниченный оператор. Применяя подобные рассуждения для $A^3$, получаем:
		
			$$\mod{(A^3x)(t)}^2 \leq M^6 \frac{(t-a)^5}{8}\norm{x}^2 $$
			И, наконец, получаем формулу для $A^n$:
			$$\mod{(A^nx)(t)}^2 \leq M^{2n} \frac{(t-a)^{2n-1}}{(2(n-1))!!}\norm{x}^2 $$
		
			Теперь можно легко показать, что выполнено необходимое условие сходимости ряда 
			$\left(\underset{n\rightarrow\infty}{\lim} a_n = 0\right)$:
			$$\norm{A^nx}^2 = \int_a^b \mod{(A^nx)(t)}^2 \,dt \leq \norm{x}^2 M^{2n} \frac{(b-a)^{2n}}{(2n)!!}$$
			{\color{gray} (Воспользовались тем фактом, что $(2n)!! = 2^n n!$.)}
			$$\norm{A^n} \leq \frac{\norm{A^nx}}{\norm{x}} =
			  M^n \left(\frac{(b-a)}{\sqrt{2}}\right)^n \frac{1}{\sqrt{n!}} \rightarrow 0$$
			  
			Так как $\norm{A^n} \rightarrow 0$, то $\exists n_0,\,\norm{A^{n_0}} < \frac{1}{2}$. Тогда
			$$\norm{I + A + \ldots + A^n + \ldots} \leq 1 + \norm{A} + \ldots + \norm{A^n} + \ldots$$
		
			Сгруппируем члены ряда (это можно сделать, поскольку все слагаемые больше нуля), учитывая, что $\norm{A^{k n_0}} \leq \norm{A^{n_0}}^{k}$:
			\begin{align*}
				1 + \underset{< \frac{1}{2}}{\norm{A^{n_0}}} + \underset{< \frac{1}{4}}{\norm{A^{2n_0}}} + \ldots &\leq 2 \\
				A + \underset{< \frac{1}{2}\norm{A}}{\norm{A^{n_0+1}}} 
				+ \underset{< \frac{1}{4}\norm{A}}{\norm{A^{2n_0+1}}} + \ldots &\leq 2 \norm{A} \\
				A^2 + \underset{< \frac{1}{2}\norm{A^2}}{\norm{A^{n_0+2}}} 
				+ \underset{< \frac{1}{4}\norm{A^2}}{\norm{A^{2n_0+2}}} + \ldots &\leq 2 \norm{A^2} \\
			\end{align*}
			Следовательно, норма этого ряда не превышает конечной суммы ограниченных слагаемых:
			$$\norm{I + A + A^2 + \ldots} \leq 2\cdot\sum_{k=0}^n\norm{A^k} \qquad n_0 < \infty$$
			
			Это в свою очередь означает, что ряд $\sum_{k=0}^\infty A^k$ сходится. Обратный оператор
			$(I-A)^{-1}$ существует --- а значит можно найти и решение. Решение единственно: это следует из
			метода последовательных приближений, находящего неподвижную точку сжимающего отображения (а она,
			как мы помним, единственна).
		\end{proof}

\end{document}
